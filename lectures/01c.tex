\documentclass[compress]{beamer}



%

\usepackage[T1]{fontenc}
%\usepackage{kmath,kerkis}
%\usepackage{fouriernc}
\usepackage[adobe-utopia]{mathdesign}
%\usepackage{arev}
\usepackage{times}
\usepackage{natbib}

\usepackage[noend]{algpseudocode}
\usepackage{xmpmulti}
\usepackage{dsfont}
\usepackage{amsmath}

\usepackage{graphicx,float,wrapfig, bbm}
\usepackage{amsfonts, comment, bbold}
\usepackage{mdwlist}
\usepackage{subfigure}
\usepackage{colortbl}
\usepackage{mathrsfs}


\usepackage{multirow}




% packages

\usepackage{amsfonts}

% environments

\newenvironment{packed_enumerate}{
  \begin{enumerate}
    \setlength{\topsep}{0pt}
    \setlength{\itemsep}{2pt}
    \setlength{\parskip}{0pt}
    \setlength{\parsep}{0pt}
}{\end{enumerate}}

\newenvironment{stepit}
 {\begin{itemize}[<+-|alert@+>]}
   {\end{itemize}}

% commands

\newcommand{\Norm}[3]{\mathcal{N}\left( #1 \g #2, #3 \right)}
\newcommand{\popshow}[2]{\only<#1->{\alert<#1>{#2}}}
\newcommand{\x}{\mathbf{x}}
\newcommand{\ex}[1]{\mbox{exp}\left\{ #1\right\} }
\newcommand{\e}[2]{\mathbb{E}_{#1}\left[ #2 \right] }
\newcommand{\g}{\, | \,}
\newcommand{\indpt}{\protect\mathpalette{\protect\independenT}{\perp}}
\def\independenT#1#2{\mathrel{\rlap{$#1#2$}\mkern2mu{#1#2}}}
\newcommand{\E}{\textrm{E}}
\newcommand{\R}{\textrm{R}}
\newcommand{\realline}{\mathbb{R}}
\newcommand{\data}{{\cal D}}
\newcommand{\loglik}{{\cal L}}
\newcommand{\grad}[2]{ \frac{\partial{#1}}{\partial#2}}
\newcommand{\dir}[1]{\mbox{Dir}(#1)}
\newcommand{\mult}[1]{\mbox{Mult}( #1)}
\newcommand{\G}[1]{\Gamma \left( \textstyle #1 \right)}
\newcommand{\ind}[1]{\mathds{1}\left[ #1 \right] }
\newcommand{\norm}[1]{\left\lVert#1\right\rVert}

\newcommand{\class}[1]{ \texttt{#1}}
\newcommand{\term}[1]{ ``#1''}
\newcommand{\tcword}[0]{ w }
\newcommand{\docsetlabeled}[0]{ D }
\newcommand{\onedoclabeled}[0]{ d }
\newcommand{\tcposindex}[0]{ i }
\newcommand{\myblue}[1]{ {\textbf #1 }}
\newcommand{\dnrm}[1]{ _{\mbox{\textsc{ #1 }}}}
\newcommand{\argmax}[0]{ \arg \max }
\newcommand{\tcjclass}[0]{c_j}
\newcommand{\maths}[1]{ {\bf #1}}

\newcommand{\fsi}[2]{
\begin{frame}[plain]
\vspace*{-1pt}
\makebox[\linewidth]{\includegraphics[width=\paperwidth]{#1}}
\begin{center}
#2
\end{center}
\end{frame}
}





% complexity
\renewcommand{\O}{\mathcal{O}}



\setbeamersize{text margin left=0.5cm}
\setbeamersize{text margin right=0.5cm}
\setbeamercolor{alert}{fg=red!75!black}


\usetheme{default}
\useinnertheme{circles}
\useoutertheme{split}
\usecolortheme{seahorse}
% \usecolortheme{dove}
% \usecolortheme{seagull}
%\usecolortheme{default}
% \usecolortheme{dolphin}
\usefonttheme{structurebold}
%\usefonttheme{serif}

\setbeamertemplate{navigation symbols}{}
\setbeamertemplate{headline}{}
\setbeamertemplate{footline}{}
\setbeamerfont{itemize/enumerate subbody}{size=\normalsize}
\setbeamerfont{itemize/enumerate subsubbody}{size=\normalsize}
\setbeamercolor{itemize item}{fg=gray}
\setbeamercolor{enumerate item}{fg=gray}
\setbeamercolor{itemize item}{fg=gray}
\setbeamercolor{itemize subitem}{fg=gray}
\setbeamercolor{item projected}{bg=gray}
\setbeamercolor{subitem projected}{bg=gray}

\newcommand{\explain}[2]{\underbrace{#2}_{\mbox{\footnotesize{#1}}}}
\newcommand\hmmax{0}
\newcommand\bmmax{0}

\newenvironment{bullets}
{\begin{itemize} \setlength{\itemsep}{10pt}}
{\end{itemize}}

\newcommand{\mygraphic}[2]{
  \begin{beamercolorbox}[colorsep*=4pt]{black math}
    \begin{center}
      \includegraphics[#1]{#2}
    \end{center}
  \end{beamercolorbox}
}

\setbeamercolor{structure}{bg=gray}
\setbeamercolor{section in head/foot}{bg=gray}
\setbeamercolor{palette primary}{bg=lightgray}


\usepackage{minted}

\usetheme[pageofpages=of,                    % String used between the current page and the
                                             % total page count.
          bullet=circle,                     % Use circles instead of squares for bullets.
          titleline=true,                    % Show a line below the frame title.
          showdate=true,                     % show the date on the title page
          alternativetitlepage=true,         % Use the fancy title page.
          titlepagelogo=culogo,              % Logo for the first page.
          % Logo for the header on first page.
          headerlogo=boulder_cs,
          ]{UCBoulder}

\usecolortheme{ucdblack}
\author{Advanced Machine Learning for NLP}


\institute[Boyd-Graber] % (optional, but mostly needed)
{Jordan Boyd-Graber}


\AtBeginSection[] % "Beamer, do the following at the start of every section"
{ \begin{frame} \frametitle{Outline} % make a frame titled "Outline"
\tableofcontents[currentsection] % show TOC and highlight current section
\end{frame} }


\newcommand{\gfx}[2]{
\begin{center}
	\includegraphics[width=#2\linewidth]{online/#1}
\end{center}
}
\title{Why Language is Hard: Structure and Predictions}
\date{Slides adapted from Liang Huang}

\begin{document}

\frame{
\titlepage
}


\begin{frame}{POS Tagging: Task Definition}

\begin{itemize}
\item Annotate each word in a sentence with a part-of-speech marker.
\item Lowest level of syntactic analysis.

\begin{scriptsize}
\begin{tabular}{cccccccccccc}
John  & saw &  the &  saw &  and & decided & to & take & it &    to &  the  & table \\
NNP & VBD & DT & NN & CC & VBD  &   TO &VB & PRP &IN &DT  &  NN
\end{tabular}
\end{scriptsize}
\end{itemize}
\end{frame}



\begin{frame}
\frametitle{Typical Features ($\phi$)}

Assume $K$ parts of speech, a lexicon size of $V$, a series of observations $\{x_1, \dots, x_N\}$, and a series of unobserved states $\{z_1, \dots, z_N\}$.

\begin{itemize}
  \item[$\pi$] Start state scores (vector of length $K$):
    $\pi_i\popshow{9}{ = \log p(z_1 = i)}$
  \item[$\theta$] Transition matrix (matrix of size $K$ by $K$):
    $\theta_{i,j}\popshow{10}{ = \log p(z_{n} = j | z_{n-1} = i)}$
  \item[$\beta$] An emission matrix (matrix of size $K$ by $V$): $\beta_{j,w} \popshow{11}{ = \log  p(x_n = w | z_n=j)}$
\end{itemize}

\only<2->{

\begin{block}{Score}
\begin{equation}
  \alert<3>{f(x, z)} \equiv \sum_i \alert<4>{w_i} \alert<5>{\phi_i(x, z)}
\end{equation}
\only<3>{Total score of hypothesis $z$ given input $x$}
\only<4>{Feature weight}
\only<5>{Feature present (binary)}
\end{block}

}

\only<6->{

Two problems: How do we move from data to algorithm?
(Estimation\only<7->{: \alert<7-8>{HMM}}) How do we move from a model and unlabled data to labeled data? (Inference)

}

\end{frame}


\begin{frame}
\frametitle{Viterbi Algorithm}

\begin{itemize}
\item Given an unobserved sequence of length $L$, $\{x_1, \dots, x_L\}$, we want to find a sequence $\{z_1 \dots z_L\}$ with the highest score.
\pause
\item It's impossible to compute $K^L$ possibilities.
\item So, we use dynamic programming to compute most likely tags for
  each token subsequence from $0$ to $t$ that ends in state $k$.
\item Memoization: fill a table of solutions of sub-problems
\item Solve larger problems by composing sub-solutions
\item Base case:
\begin{equation}
f_1(k) = \pi_k + \beta_{k, x_i}
\end{equation}
\item Recursion:
\begin{equation}
f_n(k) = \max_{\alert<3>{j}}
{\left(f_{n-1}(j)\alert<4>{+ \theta_{j,k}}\right)} \alert<5>{+ \beta_{k, x_n}}\end{equation}
\end{itemize}

\end{frame}

\begin{frame}

\begin{itemize}
\item The complexity of this is now $K^2 L$.
\item In class: example that shows why you need all $O(KL)$ table cells (garden pathing)
\item But just computing the max isn't enough.  We also have to remember where we came from.  (Breadcrumbs from best previous state.)
\begin{equation}
\Psi_{n} = \mbox{argmax}_j f_{n-1}(j) + \theta_{j,k}
\end{equation}
\pause
\item Let's do that for the sentence ``come and get it''

\end{itemize}
\end{frame}


\begin{frame}

\begin{center}
\begin{tabular}{|c|c|c|c|}
\hline
POS  & $\pi_k$ & $\beta_{k,x_1}$&  $ f_1(k)$ \\
\hline
MOD  & $\log 0.234$ & $\log 0.024$ & -5.18 \\
DET   & $\log 0.234$ & $\log 0.032$ & -4.89 \\
CONJ  & $\log 0.234$ & $\log 0.024$ & -5.18\\
N   & $\log 0.021$ & $\log 0.016$ & -7.99 \\
PREP & $\log 0.021$ & $\log 0.024$ & -7.59 \\
PRO  & $\log 0.021$ & $\log 0.016$ & -7.99 \\
V  & $\log 0.234$ & $\log 0.121$ & -3.56 \\
\hline
\multicolumn{4}{c}{{\bf come} and get it (with HMM probabilities)}
\end{tabular}

\end{center}

Why logarithms?
\begin{enumerate}
\item More interpretable than a float with lots of zeros.
\item Underflow is less of an issue
\item Generalizes to linear models (next!)
\item Addition is cheaper than multiplication
  \begin{equation}
    log(ab) = log(a) + log(b)
  \end{equation}
\end{enumerate}

\end{frame}

\begin{frame}

\begin{center}
\begin{tabular}{|c|c|c|c|}
\hline
POS  & $f_1(j) $ & \uncover<3->{$f_1(j) + \theta_{j, \mbox{CONJ}}$} & $f_2(\mbox{CONJ})$ \\
\hline
MOD  & -5.18 & \uncover<7->{-8.48} & \\
DET   &  -4.89 & \uncover<7->{-7.72} & \\
CONJ  & -5.18 & \uncover<7->{-8.47}  & \color{red}{\uncover<2-8>{???}  \uncover<11>{-6.02}}\\
N   & -7.99 & \uncover<6->{$\leq -7.99$} &  \\
PREP & -7.59 & \uncover<6->{$\leq -7.59$} & \\
PRO  & -7.99 & \uncover<6->{$\leq -7.99$} & \\
V  & -3.56 & \uncover<5->{\color<8->{green}{-5.21}} & \\
\hline
\multicolumn{4}{c}{ come {\bf and} get it}
\end{tabular}




\end{center}

\uncover<4>{
\begin{equation}
f_0(\mbox{V}) + \theta_{\mbox{V, CONJ}} = f_0(k) + \theta_{\mbox{V, CONJ}} = -3.56 + -1.65 \nonumber
\end{equation}
}

\uncover<9-10>{
\begin{equation}
\log{f_1(k)} = -5.21 + \beta_{\mbox{CONJ, and}} = \uncover<10>{-5.21-0.64}\nonumber
\end{equation}
}

\end{frame}

\begin{frame}

\begin{center}
\footnotesize{
\begin{tabular}{|c|c|c|c|c|c|c|c|}
\hline
POS  & $ f_1(k) $ & $f_2(k) $ & $b_2$ & $f_3(k) $ & $b_3$ & $f_4(k)$ & $b_4$ \\
\hline
MOD  & \color{gray}{-5.18} & \uncover<2->{\color{gray}{-0.00}} & \uncover<2->{\color{gray}{X}} & \uncover<3->{\color{gray}{-0.00}} & \uncover<3->{\color{gray}{X}} & \uncover<4->{\color{gray}{-0.00}} & \uncover<4->{\color{gray}{X}} \\
DET   & \color{gray}{-4.89} & \uncover<2->{\color{gray}{-0.00}} & \uncover<2->{\color{gray}{X}}  & \uncover<3->{\color{gray}{-0.00}} & \uncover<3->{\color{gray}{X}} & \uncover<4->{\color{gray}{-0.00}} & \uncover<4->{\color{gray}{X}} \\
CONJ  & \color{gray}{-5.18} & -6.02 & V & \uncover<3->{\color{gray}{-0.00}} & \uncover<3->{\color{gray}{X}} & \uncover<4->{\color{gray}{-0.00}} & \uncover<4->{\color{gray}{X}} \\
N   & \color{gray}{-7.99} & \uncover<2->{\color{gray}{-0.00}} & \uncover<2->{\color{gray}{X}} & \uncover<3->{\color{gray}{-0.00}} & \uncover<3->{\color{gray}{X}}  & \uncover<4->{\color{gray}{-0.00}} & \uncover<4->{\color{gray}{X}} \\
PREP & \color{gray}{-7.59} & \uncover<2->{\color{gray}{-0.00}} & \uncover<2->{\color{gray}{X}} & \uncover<3->{\color{gray}{-0.00}} & \uncover<3->{\color{gray}{X}}   & \uncover<4->{\color{gray}{-0.00}} & \uncover<4->{\color{gray}{X}} \\
PRO  & \color{gray}{-7.99} & \uncover<2->{\color{gray}{-0.00}} & \uncover<2->{\color{gray}{X}} & \uncover<3->{\color{gray}{-0.00}} & \uncover<3->{\color{gray}{X}}  &  \uncover<4->{-14.6} &  \uncover<4->{V}  \\
V  & -3.56 & \uncover<2->{\color{gray}{-0.00}} & \uncover<2->{\color{gray}{X}}  &  \uncover<3->{-9.03} &  \uncover<3->{CONJ} & \uncover<4->{\color{gray}{-0.00}} & \uncover<4->{\color{gray}{X}} \\
\hline
WORD & come & \multicolumn{2}{c|}{and} & \multicolumn{2}{c|}{get} & \multicolumn{2}{c|}{it} \\
\hline
\end{tabular}}
\end{center}

\end{frame}

\end{document}
