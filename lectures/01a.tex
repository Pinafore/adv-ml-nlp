\documentclass[compress]{beamer}



%

\usepackage[T1]{fontenc}
%\usepackage{kmath,kerkis}
%\usepackage{fouriernc}
\usepackage[adobe-utopia]{mathdesign}
%\usepackage{arev}
\usepackage{times}
\usepackage{natbib}

\usepackage[noend]{algpseudocode}
\usepackage{xmpmulti}
\usepackage{dsfont}
\usepackage{amsmath}

\usepackage{graphicx,float,wrapfig, bbm}
\usepackage{amsfonts, comment, bbold}
\usepackage{mdwlist}
\usepackage{subfigure}
\usepackage{colortbl}
\usepackage{mathrsfs}


\usepackage{multirow}




% packages

\usepackage{amsfonts}

% environments

\newenvironment{packed_enumerate}{
  \begin{enumerate}
    \setlength{\topsep}{0pt}
    \setlength{\itemsep}{2pt}
    \setlength{\parskip}{0pt}
    \setlength{\parsep}{0pt}
}{\end{enumerate}}

\newenvironment{stepit}
 {\begin{itemize}[<+-|alert@+>]}
   {\end{itemize}}

% commands

\newcommand{\Norm}[3]{\mathcal{N}\left( #1 \g #2, #3 \right)}
\newcommand{\popshow}[2]{\only<#1->{\alert<#1>{#2}}}
\newcommand{\x}{\mathbf{x}}
\newcommand{\ex}[1]{\mbox{exp}\left\{ #1\right\} }
\newcommand{\e}[2]{\mathbb{E}_{#1}\left[ #2 \right] }
\newcommand{\g}{\, | \,}
\newcommand{\indpt}{\protect\mathpalette{\protect\independenT}{\perp}}
\def\independenT#1#2{\mathrel{\rlap{$#1#2$}\mkern2mu{#1#2}}}
\newcommand{\E}{\textrm{E}}
\newcommand{\R}{\textrm{R}}
\newcommand{\realline}{\mathbb{R}}
\newcommand{\data}{{\cal D}}
\newcommand{\loglik}{{\cal L}}
\newcommand{\grad}[2]{ \frac{\partial{#1}}{\partial#2}}
\newcommand{\dir}[1]{\mbox{Dir}(#1)}
\newcommand{\mult}[1]{\mbox{Mult}( #1)}
\newcommand{\G}[1]{\Gamma \left( \textstyle #1 \right)}
\newcommand{\ind}[1]{\mathds{1}\left[ #1 \right] }
\newcommand{\norm}[1]{\left\lVert#1\right\rVert}

\newcommand{\class}[1]{ \texttt{#1}}
\newcommand{\term}[1]{ ``#1''}
\newcommand{\tcword}[0]{ w }
\newcommand{\docsetlabeled}[0]{ D }
\newcommand{\onedoclabeled}[0]{ d }
\newcommand{\tcposindex}[0]{ i }
\newcommand{\myblue}[1]{ {\textbf #1 }}
\newcommand{\dnrm}[1]{ _{\mbox{\textsc{ #1 }}}}
\newcommand{\argmax}[0]{ \arg \max }
\newcommand{\tcjclass}[0]{c_j}
\newcommand{\maths}[1]{ {\bf #1}}

\newcommand{\fsi}[2]{
\begin{frame}[plain]
\vspace*{-1pt}
\makebox[\linewidth]{\includegraphics[width=\paperwidth]{#1}}
\begin{center}
#2
\end{center}
\end{frame}
}





% complexity
\renewcommand{\O}{\mathcal{O}}



\setbeamersize{text margin left=0.5cm}
\setbeamersize{text margin right=0.5cm}
\setbeamercolor{alert}{fg=red!75!black}


\usetheme{default}
\useinnertheme{circles}
\useoutertheme{split}
\usecolortheme{seahorse}
% \usecolortheme{dove}
% \usecolortheme{seagull}
%\usecolortheme{default}
% \usecolortheme{dolphin}
\usefonttheme{structurebold}
%\usefonttheme{serif}

\setbeamertemplate{navigation symbols}{}
\setbeamertemplate{headline}{}
\setbeamertemplate{footline}{}
\setbeamerfont{itemize/enumerate subbody}{size=\normalsize}
\setbeamerfont{itemize/enumerate subsubbody}{size=\normalsize}
\setbeamercolor{itemize item}{fg=gray}
\setbeamercolor{enumerate item}{fg=gray}
\setbeamercolor{itemize item}{fg=gray}
\setbeamercolor{itemize subitem}{fg=gray}
\setbeamercolor{item projected}{bg=gray}
\setbeamercolor{subitem projected}{bg=gray}

\newcommand{\explain}[2]{\underbrace{#2}_{\mbox{\footnotesize{#1}}}}
\newcommand\hmmax{0}
\newcommand\bmmax{0}

\newenvironment{bullets}
{\begin{itemize} \setlength{\itemsep}{10pt}}
{\end{itemize}}

\newcommand{\mygraphic}[2]{
  \begin{beamercolorbox}[colorsep*=4pt]{black math}
    \begin{center}
      \includegraphics[#1]{#2}
    \end{center}
  \end{beamercolorbox}
}

\setbeamercolor{structure}{bg=gray}
\setbeamercolor{section in head/foot}{bg=gray}
\setbeamercolor{palette primary}{bg=lightgray}


\usepackage{minted}

\usetheme[pageofpages=of,                    % String used between the current page and the
                                             % total page count.
          bullet=circle,                     % Use circles instead of squares for bullets.
          titleline=true,                    % Show a line below the frame title.
          showdate=true,                     % show the date on the title page
          alternativetitlepage=true,         % Use the fancy title page.
          titlepagelogo=culogo,              % Logo for the first page.
          % Logo for the header on first page.
          headerlogo=boulder_cs,
          ]{UCBoulder}

\usecolortheme{ucdblack}
\author{Advanced Machine Learning for NLP}


\institute[Boyd-Graber] % (optional, but mostly needed)
{Jordan Boyd-Graber}


\AtBeginSection[] % "Beamer, do the following at the start of every section"
{ \begin{frame} \frametitle{Outline} % make a frame titled "Outline"
\tableofcontents[currentsection] % show TOC and highlight current section
\end{frame} }


\newcommand{\gfx}[2]{
\begin{center}
	\includegraphics[width=#2\linewidth]{online/#1}
\end{center}
}
\title{Why Language is Hard: Structure and Predictions}
\date{Slides adapted from Liang Huang}

\begin{document}

\frame{
\titlepage
}


\section{Perceptron Algorithm}

\begin{frame}{Perceptron Algorithm}

	\begin{itemize}
		\item Online algorithm for classification
		\item Very similar to logistic regression (but 0/1 loss)
		\item But what can we prove?
	\end{itemize}

\end{frame}

\begin{frame}{$k$-means}
\begin{algorithmic}[1]
\State $\vec w_1 \gets \vec 0$
 \For{$t \leftarrow 1 \dots T$}
  \State Receive $x_t$
  \State $\hat y_t \gets $ sgn$(\vec w_t \cdot \vec x_t)$
  \State Receive $y_t$
  \If{$\hat y_t \not = y_t$}
   \State $\vec w_{t+1} \gets \vec w_t + y_t \vec x_t $
   \Else
   \State $\vec w_{t+1} \gets w_t$
   \EndIf
 \EndFor
\Return $w_{T+1}$
\end{algorithmic}


\end{frame}


\begin{frame}{Objective Function}

	\begin{itemize}
		\item Optimizes
		\begin{equation}
			\frac{1}{T} \sum_t \max \left( 0, -y_t (\vec w \cdot x_t)\right)
		\end{equation}
		\item Convex but not differentiable
	\end{itemize}

\end{frame}

\begin{frame}{Margin and Errors}


\begin{columns}
	\column{.5\linewidth}
		\only<1>{\gfx{margin}{.8}}
		\only<2>{\gfx{error}{.8}}
	\column{.5\linewidth}

	\begin{itemize}
		\item If there's a good margin $\rho$, you'll converge quickly
		\pause
		\item Whenever you se an error, you move the classifier to get it right
		\item Convergence only possible if data are separable
	\end{itemize}
\end{columns}

\end{frame}


\begin{frame}{How many errors does Perceptron make?}

	\begin{itemize}
		\item If your data are in a $R$ ball and there is a margin
		\begin{equation}
		\rho \leq \frac{y_t (\vec v \cdot \vec x_t)}{||v||}
		\end{equation}
		for some $\vec v$, then the number of mistakes is bounded by $R^2/\rho^2$
		\item The places where you make an error are support vectors
		\item Convergence can be slow for small margins
	\end{itemize}

\end{frame}


\section{Online Perceptron for Structure Learning}


\begin{frame}{Binary to Structure}

	\only<1>{\gfx{bin_to_struc_0}{1.0}}
	\only<2>{\gfx{bin_to_struc_1}{1.0}}
	\only<3>{\gfx{bin_to_struc_2}{1.0}}

\end{frame}



\begin{frame}{Generic Perceptron}

\begin{itemize}

\item perceptron is the simplest machine learning algorithm
\item online-learning: one example at a time
\item learning by doing
\begin{itemize}
\item find the best output under the current weights
\item update weights at mistakes
\end{itemize}

\end{itemize}

\end{frame}

\begin{frame}{Structured Perceptron}

\gfx{struc_perceptron}{1.0}

\end{frame}


\begin{frame}{Perceptron Algorithm}

\gfx{perceptron_algorithm}{1.0}

\end{frame}


\begin{frame}{POS Example}

\gfx{pos_example}{1.0}

\end{frame}

\begin{frame}{What must be true?}

\begin{itemize}
	\item Finding highest scoring structure must be really fast (you'll do it often)
	\item Requires some sort of dynamic programming algorithm
	\item For tagging: features must be local to $y$ (but can be global to $x$)
\end{itemize}

\gfx{feature_scope}{1.0}

\end{frame}

\begin{frame}{Averaging is Good}

	\only<1>{\gfx{averaged_perceptron}{1.0}}
	\only<2>{\gfx{averaged_results}{.7}}

\end{frame}


\begin{frame}{Smoothing}
	\begin{itemize}
		\item Must include subset templates for features
		\item For example, if you have feature $(t_0, w_0, w_{-1})$, you must also have
		\begin{itemize}
			\item $(t_0, w_0)$; $(t_0, w_{-1})$; $(w_0, w_{-1})$
		\end{itemize}

	\end{itemize}
\end{frame}

\end{document}
