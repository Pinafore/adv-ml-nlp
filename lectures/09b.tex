

\documentclass[compress]{beamer}
\usefonttheme{professionalfonts}



%

\usepackage[T1]{fontenc}
%\usepackage{kmath,kerkis}
%\usepackage{fouriernc}
\usepackage[adobe-utopia]{mathdesign}
%\usepackage{arev}
\usepackage{times}
\usepackage{natbib}

\usepackage[noend]{algpseudocode}
\usepackage{xmpmulti}
\usepackage{dsfont}
\usepackage{amsmath}

\usepackage{graphicx,float,wrapfig, bbm}
\usepackage{amsfonts, comment, bbold}
\usepackage{mdwlist}
\usepackage{subfigure}
\usepackage{colortbl}
\usepackage{mathrsfs}


\usepackage{multirow}




% packages

\usepackage{amsfonts}

% environments

\newenvironment{packed_enumerate}{
  \begin{enumerate}
    \setlength{\topsep}{0pt}
    \setlength{\itemsep}{2pt}
    \setlength{\parskip}{0pt}
    \setlength{\parsep}{0pt}
}{\end{enumerate}}

\newenvironment{stepit}
 {\begin{itemize}[<+-|alert@+>]}
   {\end{itemize}}

% commands

\newcommand{\Norm}[3]{\mathcal{N}\left( #1, #2, #3 \right)}
\newcommand{\popshow}[2]{\only<#1->{\alert<#1>{#2}}}
\newcommand{\x}{\mathbf{x}}
\newcommand{\ex}[1]{\mbox{exp}\left\{ #1\right\} }
\newcommand{\e}[2]{\mathbb{E}_{#1}\left[ #2 \right] }
\newcommand{\g}{\, | \,}
\newcommand{\indpt}{\protect\mathpalette{\protect\independenT}{\perp}}
\def\independenT#1#2{\mathrel{\rlap{$#1#2$}\mkern2mu{#1#2}}}
\newcommand{\E}{\textrm{E}}
\newcommand{\R}{\textrm{R}}
\newcommand{\realline}{\mathbb{R}}
\newcommand{\data}{{\cal D}}
\newcommand{\loglik}{{\cal L}}
\newcommand{\grad}[2]{ \frac{\partial{#1}}{\partial#2}}
\newcommand{\dir}[1]{\mbox{Dir}(#1)}
\newcommand{\mult}[1]{\mbox{Mult}( #1)}
\newcommand{\G}[1]{\Gamma \left( \textstyle #1 \right)}
\newcommand{\ind}[1]{\mathds{1}\left[ #1 \right] }
\newcommand{\norm}[1]{\left\lVert#1\right\rVert}

\newcommand{\class}[1]{ \texttt{#1}}
\newcommand{\term}[1]{ ``#1''}
\newcommand{\tcword}[0]{ w }
\newcommand{\docsetlabeled}[0]{ D }
\newcommand{\onedoclabeled}[0]{ d }
\newcommand{\tcposindex}[0]{ i }
\newcommand{\myblue}[1]{ {\textbf #1 }}
\newcommand{\dnrm}[1]{ _{\mbox{\textsc{ #1 }}}}
\newcommand{\argmax}[0]{ \arg \max }
\newcommand{\tcjclass}[0]{c_j}
\newcommand{\maths}[1]{ {\bf #1}}




% complexity
\renewcommand{\O}{\mathcal{O}}



\setbeamersize{text margin left=0.5cm}
\setbeamersize{text margin right=0.5cm}
\setbeamercolor{alert}{fg=red!75!black}

\usetheme{default}
\useinnertheme{circles}
\useoutertheme{split}
\usecolortheme{seahorse}
% \usecolortheme{dove}
% \usecolortheme{seagull}
%\usecolortheme{default}
% \usecolortheme{dolphin}
\usefonttheme{structurebold}
%\usefonttheme{serif}

\setbeamertemplate{navigation symbols}{}
\setbeamertemplate{headline}{}
\setbeamertemplate{footline}{}
\setbeamerfont{itemize/enumerate subbody}{size=\normalsize}
\setbeamerfont{itemize/enumerate subsubbody}{size=\normalsize}
\setbeamercolor{itemize item}{fg=gray}
\setbeamercolor{enumerate item}{fg=gray}
\setbeamercolor{itemize item}{fg=gray}
\setbeamercolor{itemize subitem}{fg=gray}
\setbeamercolor{item projected}{bg=gray}
\setbeamercolor{subitem projected}{bg=gray}


\newenvironment{bullets}
{\begin{itemize} \setlength{\itemsep}{10pt}}
{\end{itemize}}

\newcommand{\mygraphic}[2]{
  \begin{beamercolorbox}[colorsep*=4pt]{black math}
    \begin{center}
      \includegraphics[#1]{#2}
    \end{center}
  \end{beamercolorbox}
}

\setbeamercolor{structure}{bg=gray}
\setbeamercolor{section in head/foot}{bg=gray}
\setbeamercolor{palette primary}{bg=lightgray}


\usepackage{minted}

\usetheme[pageofpages=of,                    % String used between the current page and the
                                             % total page count.
          bullet=circle,                     % Use circles instead of squares for bullets.
          titleline=true,                    % Show a line below the frame title.
          showdate=true,                     % show the date on the title page
          alternativetitlepage=true,         % Use the fancy title page.
          titlepagelogo=../../common/culogo,              % Logo for the first page.
          % Logo for the header on first page.
          headerlogo=../../common/boulder_cs,
          ]{UCBoulder}

\usecolortheme{ucdblack}
\author{Introduction to Data Science Algorithms}


\institute[Boyd-Graber and Paul] % (optional, but mostly needed)
{Jordan Boyd-Graber and Michael Paul}


\AtBeginSection[] % "Beamer, do the following at the start of every section"
{ \begin{frame} \frametitle{Outline} % make a frame titled "Outline"
\tableofcontents[currentsection] % show TOC and highlight current section
\end{frame} }

\newcommand{\gfx}[2]{
\begin{center}
	\includegraphics[width=#2\linewidth]{frameworks/#1}
\end{center}
}
\title{Frameworks}
\date{Neural Networks in DyNet}

\begin{document}


\frame{\titlepage
Slides adapted from Chris Dyer, Yoav Goldberg, Graham Neubig
}

\begin{frame}{Major Players}

\begin{itemize}
\item Computation Graph
\item Expressions (nodes in the graph)
\item Parameters
\item Model (a collection of parameters)
\item Trainer
\end{itemize}

\end{frame}

\begin{frame}[fragile]{Computation Graph}

\begin{minted}[fontsize=\footnotesize]{python}
import dynet as dy

dy.renew_cg() # create a new computation graph

v1 = dy.inputVector([1,2,3,4])
v2 = dy.inputVector([5,6,7,8])
# v1 and v2 are expressions

v3 = v1 + v2
v4 = v3 * 2
v5 = v1 + 1
v6 = dy.concatenate([v1,v3,v5])
\end{minted}
\pause

\begin{minted}[fontsize=\footnotesize]{python}
>>> print(v6)
expression 5/1
>>> print(v6.npvalue())
[  1.   2.   3.   4.   6.   8.  10.  12.   2.   3.   4.   5.]
\end{minted}

\end{frame}


\begin{frame}[fragile]{Computation Graph and Expressions}

\begin{itemize}
\item Create basic expressions.
\item Combine them using operations.
\item Expressions represent symbolic computations.
\item Actual computation:

\begin{minted}[fontsize=\footnotesize]{python}
.value()
.npvalue()               #numpy value
.scalar_value()
.vec_value()             # flatten to vector
.forward()               # compute expression
\end{minted}

\end{itemize}

\end{frame}


\begin{frame}{Models and Parameters}

\begin{itemize}
  \item {\bf Parameters} are the things that we optimize over (vectors, matrices).
  \item {\bf Model} is a collection of parameters.
  \item {\bf Parameters} out-live the computation graph.
\end{itemize}

\end{frame}

\begin{frame}[fragile]{Models and Parameters}

\begin{minted}[fontsize=\footnotesize]{python}
model = dy.Model()

pW = model.add_parameters((2,4))
pb = model.add_parameters(2)
\end{minted}

\begin{minted}[fontsize=\footnotesize]{python}
dy.renew_cg()
x = dy.inputVector([1,2,3,4])
W = dy.parameter(pW) # convert params to expression
b = dy.parameter(pb) # and add to the graph

y = W * x + b
\end{minted}

\end{frame}

\begin{frame}[fragile]{Inspecting}

Let's inspect \alert<1>{$x$}, \alert<2>{$W$}, \alert<3>{$b$}, and \alert<4>{$y$}.
\pause
\begin{minted}[fontsize=\footnotesize]{python}
>>> x.value()
[1.0, 2.0, 3.0, 4.0]
\end{minted}
\pause
\begin{minted}[fontsize=\footnotesize]{python}
>>> W.value()
array([[ 0.64952731, -0.06049263,  0.90871298, -0.11073416],
       [ 0.75935686,  0.25788534, -0.98922664,  0.20040739]])
\end{minted}
\pause
\begin{minted}[fontsize=\footnotesize]{python}
>>> b.value()
[-1.5444282293319702, -0.660666823387146]
\end{minted}
\pause
\begin{minted}[fontsize=\footnotesize]{python}
>>> y.value()
[1.267316222190857, -1.5515896081924438]
\end{minted}

\end{frame}


% TODO: Look up these initializations
\begin{frame}[fragile]{Initialization}

\begin{minted}[fontsize=\footnotesize]{python}
model = dy.Model()

pW = model.add_parameters((4,4))

pW2 = model.add_parameters((4,4),
                      init=dy.GlorotInitializer())

pW3 = model.add_parameters((4,4),
                      init=dy.NormalInitializer(0,1))
\end{minted}

\begin{block}{Glorot Initialization}
  \begin{equation}
    \Norm{w_i}{0}{\frac{1}{n_{\textit{in}} + n_{\textit{out}}}}
  \end{equation}
\end{block}

\end{frame}


\begin{frame}{Trainers and Backprop}

\begin{itemize}

\item Initialize a Trainer with a given model.
\item Compute gradients by calling expr.backward() from a scalar node.
\item Call trainer.update() to update the model parameters using the gradients.

\end{itemize}

\end{frame}


\begin{frame}[fragile]{Trainers and Backprop}

\begin{minted}[fontsize=\footnotesize]{python}
model = dy.Model()

trainer = dy.SimpleSGDTrainer(model)

p_v = model.add_parameters(10)

for i in xrange(10):
    dy.renew_cg()

    v = dy.parameter(p_v)
    v2 = dy.dot_product(v,v)
    v2.forward()

    v2.backward()  # compute gradients
    trainer.update()
\end{minted}

\end{frame}

\begin{frame}[fragile]{Options for Trainers}

\begin{minted}[fontsize=\footnotesize]{python}
  dy.SimpleSGDTrainer(model,...)

  dy.MomentumSGDTrainer(model,...)

  dy.AdagradTrainer(model,...)

  dy.AdadeltaTrainer(model,...)

  dy.AdamTrainer(model,...)
\end{minted}

\end{frame}

\begin{frame}{Training with DyNet}

\begin{itemize}
\item Create model, add parameters, create trainer.
\item For each training example:
\begin{itemize}
  \item create computation graph for the loss
  \item run forward (compute the loss)
  \item run backward (compute the gradients)
  \item update parameters
\end{itemize}
\end{itemize}

\end{frame}

\begin{frame}{Multilayer Perceptron for XOR}

\begin{itemize}
  \item Model
    \begin{equation}
      \hat y = \sigma(\hat v \cdot \mbox{tanh}(U \vec x + b))
      \end{equation}
      \item Loss
        \begin{equation}
        \ell = \begin{cases} - \log \hat y & \mbox{if }y = 0\\
          - \log (1 - \hat y) & \mbox{if }y = 0\\
          \end{cases}
          \end{equation}
\end{itemize}

\end{frame}

\begin{frame}[fragile]{Imports and Data}

\begin{minted}[fontsize=\footnotesize]{python}
import dynet as dy
import random

data =[ ([0,1],0),
        ([1,0],0),
        ([0,0],1),
        ([1,1],1) ]
\end{minted}

\end{frame}


\begin{frame}[fragile]{Create Model}

\begin{minted}[fontsize=\footnotesize]{python}
model = dy.Model()
pU = model.add_parameters((4,2))
pb = model.add_parameters(4)
pv = model.add_parameters(4)

trainer = dy.SimpleSGDTrainer(model)
closs = 0.0
\end{minted}

\end{frame}


\begin{frame}[fragile]{}

\begin{minted}[fontsize=\footnotesize]{python}
for x,y in data:
   # create graph for computing loss
   dy.renew_cg()
   U = dy.parameter(pU)
   b = dy.parameter(pb)
   v = dy.parameter(pv)
   x = dy.inputVector(x)
   # predict
   yhat = dy.logistic(dy.dot_product(v,dy.tanh(U*x+b)))
   # loss
   if y == 0:
      loss = -dy.log(1 - yhat)
   elif y == 1:
      loss = -dy.log(yhat)

   closs += loss.scalar_value() # forward
   loss.backward()
   trainer.update()
\end{minted}
\pause
{\bf Important}: loss expression defines objective you're optimizing

\end{frame}

\begin{frame}{Key Points}

\begin{itemize}
  \item Create computation graph for each example.
  \item Graph is built by composing expressions.
  \item Functions that take expressions and return expressions define graph components.
\end{itemize}

\end{frame}

\begin{frame}{Word Embeddings and Lookup Parameters}

\begin{itemize}
  \item In NLP, it is very common to use feature embeddings.
  \item Each feature is represented as a $d$-dim vector.
  \item These are then summed or concatenated to form an input vector.
  \item The embeddings can be pre-trained.
  \item They are usually trained with the model.
\end{itemize}

\end{frame}

%----------------------------------
\fsi{frameworks/feat_embed}{}
%----------------------------------

\begin{frame}[fragile]{}

\begin{minted}[fontsize=\footnotesize]{python}
vocab_size = 10000
emb_dim = 200

E = model.add_lookup_parameters((vocab_size, emb_dim))

dy.renew_cg()
x = dy.lookup(E, 5)
# or
x = E[5]
# x is an expression
\end{minted}

\end{frame}
%----------------------------------

\fsi{frameworks/dan_title}{Implementing a non-trivial example \dots}

%----------------------------------

\begin{frame}[fragile]{Deep Averaging Network}

\begin{columns}
  \column{.3\linewidth}

  \begin{align*}
    w_1, &\dots, w_N \\
    & \downarrow \\
    z_0 = & \mbox{CBOW}(w_1, \dots, w_N) \\
    z_1 = & g(z_1) \\
    z_2 = & g(z_2) \\
    \hat y = & \mbox{softmax}(z_3)
  \end{align*}

  \column{.7\linewidth}

    \begin{itemize}
      \item Works about as well as more complicated models
      \item Strong baseline
      \item Key idea: Continuous Bag of Words
        \begin{equation}
          \mbox{CBOW}(w_1, \dots, w_N) = \sum_i E[w_i]
        \end{equation}
      \item Actual non-linearity doesn't matter, we'll use tanh
      \item Let's implement in DyNet
    \end{itemize}

\end{columns}
\end{frame}
%----------------------------------

\begin{frame}[fragile]{Deep Averaging Network}

\begin{columns}
  \column{.3\linewidth}

  \begin{align*}
    w_1, &\dots, w_N \\
    & \downarrow \\
    z_0 = & \alert<1>{\mbox{CBOW}(w_1, \dots, w_N)} \\
    z_1 = & \alert<2>{g(z_1)} \\
    z_2 = & \alert<3>{g(z_2)} \\
    \hat y = & \mbox{softmax}(z_3)
  \end{align*}

  \column{.7\linewidth}

\alert<1>{Encode the document}
\begin{minted}[fontsize=\footnotesize]{python}
def encode_doc(doc):
    doc = [w2i[w] for w in doc]
    embs = [E[idx] for idx in doc]
    return dy.esum(embs)
\end{minted}

\alert<2>{First Layer}
\begin{minted}[fontsize=\footnotesize]{python}
def layer1(x):
    W = dy.parameter(pW1)
    b = dy.parameter(pb1)
    return dy.tanh(W*x+b)
\end{minted}

\alert<3>{Second Layer}
\begin{minted}[fontsize=\footnotesize]{python}
def layer2(x):
    W = dy.parameter(pW2)
    b = dy.parameter(pb2)
    return dy.tanh(W*x+b)
\end{minted}

\end{columns}
\end{frame}


\begin{frame}[fragile]{Deep Averaging Network}

\begin{columns}
  \column{.3\linewidth}

  \begin{align*}
    w_1, &\dots, w_N \\
    & \downarrow \\
    z_0 = & \mbox{CBOW}(w_1, \dots, w_N) \\
    z_1 = & g(z_1) \\
    z_2 = & g(z_2) \\
    \hat y = & \mbox{softmax}(z_3)
  \end{align*}

  \column{.7\linewidth}

\alert<1>{Loss}
\begin{minted}[fontsize=\tiny]{python}
def do_loss(probs, label):
    label = label_indicator[label]
    return -dy.log(dy.pick(probs,label)) # select that index
\end{minted}

\alert<2>{Putting it all together}
\begin{minted}[fontsize=\tiny]{python}
def predict_labels(doc):
    x = encode_doc(doc)
    h = layer1(x)
    y = layer2(h)
    return dy.softmax(y)
\end{minted}

\alert<3>{Training}
\begin{minted}[fontsize=\tiny]{python}
for (doc, label) in data:
    dy.renew_cg()
    probs = predict_labels(doc)

    loss = do_loss(probs,label)
    loss.forward()
    loss.backward()
    trainer.update()
\end{minted}

\end{columns}
\end{frame}


\begin{frame}{Summary}

\begin{itemize}
  \item Computation Graph
  \item Expressions ($\approx$ nodes in the graph)
  \item Parameters, LookupParameters
  \item Model (a collection of parameters)
  \item Trainers
  \item Create a graph for each example, then
compute loss, backdrop, update.
\end{itemize}

\end{frame}

\end{document}