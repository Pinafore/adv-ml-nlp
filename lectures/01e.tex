\documentclass[compress]{beamer}



%

\usepackage[T1]{fontenc}
%\usepackage{kmath,kerkis}
%\usepackage{fouriernc}
\usepackage[adobe-utopia]{mathdesign}
%\usepackage{arev}
\usepackage{times}
\usepackage{natbib}

\usepackage[noend]{algpseudocode}
\usepackage{xmpmulti}
\usepackage{dsfont}
\usepackage{amsmath}

\usepackage{graphicx,float,wrapfig, bbm}
\usepackage{amsfonts, comment, bbold}
\usepackage{mdwlist}
\usepackage{subfigure}
\usepackage{colortbl}
\usepackage{mathrsfs}


\usepackage{multirow}




% packages

\usepackage{amsfonts}

% environments

\newenvironment{packed_enumerate}{
  \begin{enumerate}
    \setlength{\topsep}{0pt}
    \setlength{\itemsep}{2pt}
    \setlength{\parskip}{0pt}
    \setlength{\parsep}{0pt}
}{\end{enumerate}}

\newenvironment{stepit}
 {\begin{itemize}[<+-|alert@+>]}
   {\end{itemize}}

% commands

\newcommand{\Norm}[3]{\mathcal{N}\left( #1 \g #2, #3 \right)}
\newcommand{\popshow}[2]{\only<#1->{\alert<#1>{#2}}}
\newcommand{\x}{\mathbf{x}}
\newcommand{\ex}[1]{\mbox{exp}\left\{ #1\right\} }
\newcommand{\e}[2]{\mathbb{E}_{#1}\left[ #2 \right] }
\newcommand{\g}{\, | \,}
\newcommand{\indpt}{\protect\mathpalette{\protect\independenT}{\perp}}
\def\independenT#1#2{\mathrel{\rlap{$#1#2$}\mkern2mu{#1#2}}}
\newcommand{\E}{\textrm{E}}
\newcommand{\R}{\textrm{R}}
\newcommand{\realline}{\mathbb{R}}
\newcommand{\data}{{\cal D}}
\newcommand{\loglik}{{\cal L}}
\newcommand{\grad}[2]{ \frac{\partial{#1}}{\partial#2}}
\newcommand{\dir}[1]{\mbox{Dir}(#1)}
\newcommand{\mult}[1]{\mbox{Mult}( #1)}
\newcommand{\G}[1]{\Gamma \left( \textstyle #1 \right)}
\newcommand{\ind}[1]{\mathds{1}\left[ #1 \right] }
\newcommand{\norm}[1]{\left\lVert#1\right\rVert}

\newcommand{\class}[1]{ \texttt{#1}}
\newcommand{\term}[1]{ ``#1''}
\newcommand{\tcword}[0]{ w }
\newcommand{\docsetlabeled}[0]{ D }
\newcommand{\onedoclabeled}[0]{ d }
\newcommand{\tcposindex}[0]{ i }
\newcommand{\myblue}[1]{ {\textbf #1 }}
\newcommand{\dnrm}[1]{ _{\mbox{\textsc{ #1 }}}}
\newcommand{\argmax}[0]{ \arg \max }
\newcommand{\tcjclass}[0]{c_j}
\newcommand{\maths}[1]{ {\bf #1}}

\newcommand{\fsi}[2]{
\begin{frame}[plain]
\vspace*{-1pt}
\makebox[\linewidth]{\includegraphics[width=\paperwidth]{#1}}
\begin{center}
#2
\end{center}
\end{frame}
}





% complexity
\renewcommand{\O}{\mathcal{O}}



\setbeamersize{text margin left=0.5cm}
\setbeamersize{text margin right=0.5cm}
\setbeamercolor{alert}{fg=red!75!black}


\usetheme{default}
\useinnertheme{circles}
\useoutertheme{split}
\usecolortheme{seahorse}
% \usecolortheme{dove}
% \usecolortheme{seagull}
%\usecolortheme{default}
% \usecolortheme{dolphin}
\usefonttheme{structurebold}
%\usefonttheme{serif}

\setbeamertemplate{navigation symbols}{}
\setbeamertemplate{headline}{}
\setbeamertemplate{footline}{}
\setbeamerfont{itemize/enumerate subbody}{size=\normalsize}
\setbeamerfont{itemize/enumerate subsubbody}{size=\normalsize}
\setbeamercolor{itemize item}{fg=gray}
\setbeamercolor{enumerate item}{fg=gray}
\setbeamercolor{itemize item}{fg=gray}
\setbeamercolor{itemize subitem}{fg=gray}
\setbeamercolor{item projected}{bg=gray}
\setbeamercolor{subitem projected}{bg=gray}

\newcommand{\explain}[2]{\underbrace{#2}_{\mbox{\footnotesize{#1}}}}
\newcommand\hmmax{0}
\newcommand\bmmax{0}

\newenvironment{bullets}
{\begin{itemize} \setlength{\itemsep}{10pt}}
{\end{itemize}}

\newcommand{\mygraphic}[2]{
  \begin{beamercolorbox}[colorsep*=4pt]{black math}
    \begin{center}
      \includegraphics[#1]{#2}
    \end{center}
  \end{beamercolorbox}
}

\setbeamercolor{structure}{bg=gray}
\setbeamercolor{section in head/foot}{bg=gray}
\setbeamercolor{palette primary}{bg=lightgray}


\usepackage{minted}

\usetheme[pageofpages=of,                    % String used between the current page and the
                                             % total page count.
          bullet=circle,                     % Use circles instead of squares for bullets.
          titleline=true,                    % Show a line below the frame title.
          showdate=true,                     % show the date on the title page
          alternativetitlepage=true,         % Use the fancy title page.
          titlepagelogo=culogo,              % Logo for the first page.
          % Logo for the header on first page.
          headerlogo=boulder_cs,
          ]{UCBoulder}

\usecolortheme{ucdblack}
\author{Advanced Machine Learning for NLP}


\institute[Boyd-Graber] % (optional, but mostly needed)
{Jordan Boyd-Graber}


\AtBeginSection[] % "Beamer, do the following at the start of every section"
{ \begin{frame} \frametitle{Outline} % make a frame titled "Outline"
\tableofcontents[currentsection] % show TOC and highlight current section
\end{frame} }


\newcommand{\gfx}[2]{
\begin{center}
	\includegraphics[width=#2\linewidth]{online/#1}
\end{center}
}
\title{Inexact Search is ``Good Enough''}
\date{Mathematical Treatment}

\begin{document}

\frame{
\titlepage
}

\begin{frame}{Preliminaries: algorithm, separability}

  \begin{itemize}
    \item Structured perceptron maintains set of ``wrong features''
\begin{equation}
\Delta \vec \Phi (x, y, z) \equiv \vec \Phi(x, y) - \vec \Phi(x, z)
\end{equation}
    \item Structured perceptron updates weights with
\begin{equation}
\vec w \leftarrow \vec w + \Delta \vec \Phi (x, y, z)
\end{equation}
\item Dataset $D$ is linearly separable under features $\Phi$ with margin $\delta$ if
\begin{equation}
\vec u \cdot \Delta \vec \Phi (x, y, z) \geq \delta \hphantom{\dots} \forall x, y, z \in D
\end{equation}
given some oracle unit vector $u$.
  \end{itemize}

\end{frame}


\begin{frame}{Violations vs. Errors}

	\begin{itemize}
                \item It may be difficult to find the highest scoring
                  hypothesis
                 \item It's okay as long as inference finds a {\bf
                     violation}
                   \begin{equation}
                     \vec w \cdot \Delta \vec \Phi(x, y, z) \leq 0
                   \end{equation}
                 \item This means that $y$ might not be answer
                   algorithm gives (i.e., wrong)
	\end{itemize}

\end{frame}

\begin{frame}{Limited number of mistakes}

\begin{itemize}
  \item Define diameter $R$ as
    \begin{equation}
      R = \max_{(x,y,z)} || \Delta \vec \Phi (x, y, z) ||
    \end{equation}
    \pause
    \item Weight vector $\vec w$ grows with each error
    \item We can prove that $|| \vec w ||$ can't get too big
    \item And thus, algorithm can only run for limited number of
      iterations $k$ where it updates weights
    \item Indeed, we'll bound it from two directions
      \begin{equation}
        k^2 \delta^2 \leq ||w^{(k + 1)}||^2 \leq k R^2
      \end{equation}
\end{itemize}

\end{frame}


\begin{frame}{Lower Bound}
  \begin{block}{Lower Bound}
    \begin{center}
    $k^2 \delta^2 \leq ||w^{(k + 1)} ||^2$
    \end{center}
  \end{block}

  \begin{align}
    \only<2-5>{\vec w^{(k+1)} = & w^{(k)} + \Delta \vec \Phi(x, y, z)
    \\}
    \only<3-5>{\vec u \cdot \vec w^{(k+1)} = & \vec u \cdot w^{(k)} +
                                               \alert<4>{\vec u \cdot \Delta
                                        \vec \Phi(x, y, z)} \\}
    \only<4->{\vec u \cdot \vec w^{(k+1)} \geq & \vec u \cdot w^{(k)} + \alert<4>{\delta} }
   \end{align}

   \begin{center}
     \only<2>{Update equation}
     \only<3>{Multiply both sides by $\vec u$}
     \only<4>{Definition of margin}
     \only<5->{By induction, $\vec u \cdot \vec w^{(k+1)} \geq
       k\delta$ (Base case: $\vec w^0
       = \vec 0$) }
   \end{center}

\only<6->{
   \begin{align}
     \only<6->{ \alert<7>{|| \vec u ||} \hphantom{,} || \vec w^{(k+1)} || \geq \vec u \cdot \vec w
     \geq & k \delta }
     \only<7->{\\ || \vec w^{(k+1)} || \geq & k \delta }
     \only<8->{\\ || \vec w^{(k+1)} ||^{\alert<8>{2}} \geq  & k ^{\alert<8>{2}} \delta ^{\alert<8>{2}} }
   \end{align}
}

   \begin{center}
     \only<6>{ For any vectors, $|| \vec a || \hphantom{,} || \vec b || \geq a \cdot b$ }
     \only<7>{ $\vec u$ is a unit vector}
     \only<8>{ Square both sides, and we're done! }
   \end{center}


\end{frame}



\begin{frame}{Upper Bound}

  \begin{block}{Upper Bound}
    \begin{equation}
    || \vec w^{(k+1)} ||^2 \leq k R^2
    \end{equation}
  \end{block}

  \begin{align}
\only<2->{    || \vec w^{(k+1)} ||^2 = & || \vec w^{(k)} + \Delta \vec \Phi(x,
                               y, z) ||^2 }
\only<3->{ \\ || \vec w^{(k+1)} ||^2  = & || \vec w^{(k)} ||^2 + \alert<4>{||\Delta \vec \Phi(x,
                               y, z) ||}^2 + 2 w^{(k)} \cdot \Delta \vec \Phi(x,
                               y, z)  }
\only<4->{ \\ || \vec w^{(k+1)} ||^2  \leq & || \vec w^{(k)} ||^2 + R^2 + 2 \alert<5>{w^{(k)} \cdot \Delta \vec \Phi(x,
                               y, z) } }
\only<5->{ \\ || \vec w^{(k+1)} ||^2  \leq & || \vec w^{(k)} ||^2 +
                                             R^2 + 0}
\only<6->{ \\ || \vec w^{(k+1)} ||^2  \leq & kR^2}
    \end{align}

\begin{center}
    \only<2>{ Update rule}
    \only<3>{ Law of cosines }
    \only<4>{ Definition of diameter }
    \only<5>{ If violation }
    \only<6>{Induction!}
\end{center}
\end{frame}


\begin{frame}{Putting it together}

  \begin{itemize}
    \item Sandwich:
      \begin{equation}
k^2 \delta^2 \leq ||w^{(k + 1)} ||^2 \leq k R^2
\end{equation}
\pause
\item Solve for $k$:
  \begin{equation}
    k \leq \frac{R^2}{\delta^2}
    \end{equation}
    \pause
    \item What does this mean?
      \pause
      \item Limited number of errors (updates)
        \begin{itemize}
          \item Larger diameter increases errors (worst possible
            mistake)
          \item Larger margin decreases errors (bigger separation from
            wrong answer)
        \end{itemize}
       \item Finding the largest violation wrong answer is best (but
         any violation okay)
  \end{itemize}

\end{frame}


\begin{frame}{In Practice}

  Harder the search space, the more max violation helps

  \gfx{max_violation}{.8}

\end{frame}

\end{document}
