
\documentclass[compress]{beamer}
\usefonttheme{professionalfonts}



%

\usepackage[T1]{fontenc}
%\usepackage{kmath,kerkis}
%\usepackage{fouriernc}
\usepackage[adobe-utopia]{mathdesign}
%\usepackage{arev}
\usepackage{times}
\usepackage{natbib}

\usepackage[noend]{algpseudocode}
\usepackage{xmpmulti}
\usepackage{dsfont}
\usepackage{amsmath}

\usepackage{graphicx,float,wrapfig, bbm}
\usepackage{amsfonts, comment, bbold}
\usepackage{mdwlist}
\usepackage{subfigure}
\usepackage{colortbl}
\usepackage{mathrsfs}


\usepackage{multirow}




% packages

\usepackage{amsfonts}

% environments

\newenvironment{packed_enumerate}{
  \begin{enumerate}
    \setlength{\topsep}{0pt}
    \setlength{\itemsep}{2pt}
    \setlength{\parskip}{0pt}
    \setlength{\parsep}{0pt}
}{\end{enumerate}}

\newenvironment{stepit}
 {\begin{itemize}[<+-|alert@+>]}
   {\end{itemize}}

% commands

\newcommand{\Norm}[3]{\mathcal{N}\left( #1 \g #2, #3 \right)}
\newcommand{\popshow}[2]{\only<#1->{\alert<#1>{#2}}}
\newcommand{\x}{\mathbf{x}}
\newcommand{\ex}[1]{\mbox{exp}\left\{ #1\right\} }
\newcommand{\e}[2]{\mathbb{E}_{#1}\left[ #2 \right] }
\newcommand{\g}{\, | \,}
\newcommand{\indpt}{\protect\mathpalette{\protect\independenT}{\perp}}
\def\independenT#1#2{\mathrel{\rlap{$#1#2$}\mkern2mu{#1#2}}}
\newcommand{\E}{\textrm{E}}
\newcommand{\R}{\textrm{R}}
\newcommand{\realline}{\mathbb{R}}
\newcommand{\data}{{\cal D}}
\newcommand{\loglik}{{\cal L}}
\newcommand{\grad}[2]{ \frac{\partial{#1}}{\partial#2}}
\newcommand{\dir}[1]{\mbox{Dir}(#1)}
\newcommand{\mult}[1]{\mbox{Mult}( #1)}
\newcommand{\G}[1]{\Gamma \left( \textstyle #1 \right)}
\newcommand{\ind}[1]{\mathds{1}\left[ #1 \right] }
\newcommand{\norm}[1]{\left\lVert#1\right\rVert}

\newcommand{\class}[1]{ \texttt{#1}}
\newcommand{\term}[1]{ ``#1''}
\newcommand{\tcword}[0]{ w }
\newcommand{\docsetlabeled}[0]{ D }
\newcommand{\onedoclabeled}[0]{ d }
\newcommand{\tcposindex}[0]{ i }
\newcommand{\myblue}[1]{ {\textbf #1 }}
\newcommand{\dnrm}[1]{ _{\mbox{\textsc{ #1 }}}}
\newcommand{\argmax}[0]{ \arg \max }
\newcommand{\tcjclass}[0]{c_j}
\newcommand{\maths}[1]{ {\bf #1}}

\newcommand{\fsi}[2]{
\begin{frame}[plain]
\vspace*{-1pt}
\makebox[\linewidth]{\includegraphics[width=\paperwidth]{#1}}
\begin{center}
#2
\end{center}
\end{frame}
}





% complexity
\renewcommand{\O}{\mathcal{O}}



\setbeamersize{text margin left=0.5cm}
\setbeamersize{text margin right=0.5cm}
\setbeamercolor{alert}{fg=red!75!black}


\usetheme{default}
\useinnertheme{circles}
\useoutertheme{split}
\usecolortheme{seahorse}
% \usecolortheme{dove}
% \usecolortheme{seagull}
%\usecolortheme{default}
% \usecolortheme{dolphin}
\usefonttheme{structurebold}
%\usefonttheme{serif}

\setbeamertemplate{navigation symbols}{}
\setbeamertemplate{headline}{}
\setbeamertemplate{footline}{}
\setbeamerfont{itemize/enumerate subbody}{size=\normalsize}
\setbeamerfont{itemize/enumerate subsubbody}{size=\normalsize}
\setbeamercolor{itemize item}{fg=gray}
\setbeamercolor{enumerate item}{fg=gray}
\setbeamercolor{itemize item}{fg=gray}
\setbeamercolor{itemize subitem}{fg=gray}
\setbeamercolor{item projected}{bg=gray}
\setbeamercolor{subitem projected}{bg=gray}

\newcommand{\explain}[2]{\underbrace{#2}_{\mbox{\footnotesize{#1}}}}
\newcommand\hmmax{0}
\newcommand\bmmax{0}

\newenvironment{bullets}
{\begin{itemize} \setlength{\itemsep}{10pt}}
{\end{itemize}}

\newcommand{\mygraphic}[2]{
  \begin{beamercolorbox}[colorsep*=4pt]{black math}
    \begin{center}
      \includegraphics[#1]{#2}
    \end{center}
  \end{beamercolorbox}
}

\setbeamercolor{structure}{bg=gray}
\setbeamercolor{section in head/foot}{bg=gray}
\setbeamercolor{palette primary}{bg=lightgray}


\usepackage{minted}

\usetheme[pageofpages=of,                    % String used between the current page and the
                                             % total page count.
          bullet=circle,                     % Use circles instead of squares for bullets.
          titleline=true,                    % Show a line below the frame title.
          showdate=true,                     % show the date on the title page
          alternativetitlepage=true,         % Use the fancy title page.
          titlepagelogo=culogo,              % Logo for the first page.
          % Logo for the header on first page.
          headerlogo=boulder_cs,
          ]{UCBoulder}

\usecolortheme{ucdblack}
\author{Advanced Machine Learning for NLP}


\institute[Boyd-Graber] % (optional, but mostly needed)
{Jordan Boyd-Graber}


\AtBeginSection[] % "Beamer, do the following at the start of every section"
{ \begin{frame} \frametitle{Outline} % make a frame titled "Outline"
\tableofcontents[currentsection] % show TOC and highlight current section
\end{frame} }


\newcommand{\gfx}[2]{
\begin{center}
	\includegraphics[width=#2\linewidth]{rl/#1}
\end{center}
}
\title{Reinforcement Learning for NLP}
\date{Reinforcement Overview, Policy Gradient}

\usepackage{dependency/linkage6}

\begin{document}

\frame{\titlepage
\tiny Adapted from slides by David Silver, Pieter Abbeel, and John Schulman
}

\begin{frame}

  \begin{itemize}
    \item I used to say that RL wasn't used in NLP \dots
    \item Now it's all over the place
    \item Part of much of ML hype
    \item But what is reinforcement learning?
      \pause
      \begin{itemize}
\item RL is a general-purpose framework for decision-making
\item RL is for an agent with the capacity to act
\item  Each action influences the agent's future state
\item  Success is measured by a scalar reward signal
\item  Goal: select actions to maximise future reward
      \end{itemize}
  \end{itemize}

\end{frame}

\begin{frame}

\begin{columns}

  \column{.5\linewidth}
  \gfx{big_picture}{1.0}
  \column{.5\linewidth}
  \begin{itemize}
\item At each step $t$ the agent:
\begin{itemize}
\item Executes action $a_t$
\item Receives observation $o_t$
\item Receives scalar reward $r_t$
\end{itemize}
\item The environment:
\begin{itemize}
\item Receives action $a_t$
\item Emits observation $o_{t+1}$
\item Emits scalar reward $r_{t+1}$
    \end{itemize}
    \end{itemize}
\end{columns}

\end{frame}

\begin{frame}{Example}

\begin{tabular}{ccc}
\hline
  & QA & MT \\
\hline
{\bf State} & Words Seen & Foreign Words Seen \\
{\bf Reward} & Answer Accuracy & Translation Quality \\
{\bf Actions} & Answer / Wait & Translate / Wait \\
\hline
\end{tabular}


\end{frame}


\begin{frame}{State}

  \begin{itemize}
\item Experience is a sequence of observations, actions, rewards
\begin{equation}
o_1,r_1, a_1, \dots, a_{t−1}, o_t, r_t
\end{equation}
\item The state is a summary of experience
\begin{equation}
s_t = f (o_1, r_1, a_1, \dots, a_{t−1}, o_t, r_t)
\end{equation}
\item In a fully observed environment
\begin{equation}
s_t = f (o_t)
\end{equation}
  \end{itemize}

\end{frame}



\begin{frame}{What makes an RL agent?}

  \begin{itemize}
\item Policy: agent's behaviour function
\item Value function: how good is each state and/or action
\item Model: agent's representation of the environment
  \end{itemize}

\end{frame}



\begin{frame}{Policy}

  \begin{itemize}
\item A policy is the agent's behavior
  \begin{itemize}
\item  It is a map from state to action:
\item  Deterministic policy: $a = \pi(s)$
\item Stochastic policy: $\pi(a \g s) = p(a \g s)$
  \end{itemize}
  \end{itemize}
\end{frame}



\begin{frame}{Value Function}

  \begin{itemize}
\item A value function is a prediction of future reward: ``How much reward will I get from action a in state s?''
\item $Q$-value function gives expected total reward
\begin{itemize}
\item from state $s$ and action $a$
\item under policy $\pi$
\item with discount factor $\gamma$ (future rewards mean less than immediate)
\begin{equation}
Q^{\pi}(s, a) = \e{}{r_{t+1} + \gamma r_{t+2} + \gamma^2 r_{t+3} +
  \dots \g s, a}
\end{equation}
  \end{itemize}
  \end{itemize}

\end{frame}


\begin{frame}{A Value Function is Great!}

  \begin{itemize}
    \item  An optimal value function is the maximum achievable value
      \begin{equation}
Q^{*}(s, a) = \max_\pi Q^{\pi}(s, a) = Q^{\pi^*}(s, a)
\end{equation}
    \item If you know the value function, you can derive policy
      \begin{equation}
\pi^{*} = \arg \max_a Q(s, a)
      \end{equation}

  \end{itemize}

\end{frame}


\begin{frame}{Approaches to RL}


Value-based RL
\begin{itemize}
\item Estimate the optimal value function $Q^∗(s, a)$
\item This is the maximum value achievable under any policy
\end{itemize}
Policy-based RL
\begin{itemize}
\item Search directly for the optimal policy $\pi^*$
\item This is the policy achieving maximum future reward
\end{itemize}
Model-based RL
\begin{itemize}
\item Build a model of the environment
\item Plan (e.g. by lookahead) using model
\end{itemize}


\end{frame}

\begin{frame}{Deep $Q$ Learning}

\begin{itemize}
\item Optimal $Q$-values should obey equation
\begin{equation}
Q^*(s, a) = \e{s'}{r + \gamma Q(s', a') \g s, a}
\end{equation}
\item Treat as regression problem
\item Minimize: $\left(r + \gamma \max_a Q(s', a', \vec w) - Q(s, a,
    \vec w) \right)^2$
\item Converges to $Q^∗$ using table lookup representation
\item But diverges using neural networks due to:
\begin{itemize}
\item Correlations between samples
\item Non-stationary targets
\end{itemize}
\end{itemize}
\end{frame}

\begin{frame}{Deep RL in Atari}

  \gfx{atari}{.9}

\end{frame}

\begin{frame}{DQN in Atari}

\gfx{atari_net}{.8}
\begin{itemize}
\item End-to-end learning of values $Q(s, a)$ from pixels $s$
\item Input state s is stack of raw pixels from last four frames
\item Output is $Q(s, a)$ for 18 joystick/button positions
\item Reward is change in score for that step
\end{itemize}

\end{frame}


\begin{frame}{Atari Results}
  \gfx{atari_results}{.9}
\end{frame}


\begin{frame}{Policy-Based RL}

\begin{itemize}
  \item Advantages:
    \begin{itemize}
      \item Better convergence properties
      \item Effective in high-dimensional or continuous action spaces
        \item Can learn stochastic policies
          \end{itemize}
\item Disadvantages:
  \begin{itemize}
    \item Typically converge to a local rather than global optimum
      \item Evaluating a policy is typically inefficient and high
        variance
\end{itemize}
\end{itemize}
\end{frame}

\begin{frame}{Optimal Policies Sometimes Stochastic}

  \only<1>{\gfx{rps}{.8}}
  \only<2>{
    \gfx{aliased_gw}{.8}
    (Cannot distinguish gray states)
}


  \only<3-4>{Deterministic
    \gfx{aliased_gw_det}{.8}
    (Cannot distinguish gray states)

    \only<4>{Value-based RL learns near deterministic policy!}

    }
  \only<5>{Stochastic
    \gfx{aliased_gw_rand}{.8}
    (Cannot distinguish gray states, so flip a coin!)
    }
\end{frame}

\begin{frame}{Likelihood Ratio Policy Gradient}

Let $\tau$ be state-action $s_0, u_0, \dots, s_H, u_H$.  Utility of
policy $\pi$ parametrized by $\theta$ is
\begin{equation}
  U(\theta) = \e{\pi_\theta, U}{\sum_{t}^{H} R(s_t, u_t); \pi_\theta}
  = \sum_{tau} P(\tau; \theta) R(\tau).
\end{equation}
Our goal is to find $\theta$:
\begin{equation}
  \max_\theta U(\theta) = \max_\theta \sum_t p(\tau; \theta) R(\tau)
\end{equation}

\end{frame}

\begin{frame}{Likelihood Ratio Policy Gradient}
\begin{equation}
\sum_t p(\tau; \theta) R(\tau)
\end{equation}
Taking the gradient wrt $\theta$:
\begin{align}
  \only<2-4>{\nabla_\theta U(\theta) = &\sum_\tau \alert<2>{R(\tau) \frac{P(\tau;
  \theta)}{P(\tau; \theta)}} \nabla_\theta P(\tau; \theta) \\}
  \only<3-4>{ = &\sum_\tau P(\tau; \theta) \frac{\nabla_\theta P(\tau;
  \theta)}{P(\tau; \theta)} R(\tau) \\ }
  \only<4->{ = &\sum_\tau P(\tau; \theta) \nabla_\theta \left[ \log
  P(\tau; \theta) \right] R(\tau)}
\end{align}
\only<2>{Move differentiation inside sum (ignore $R(\tau)$ and then
  add in term that cancels out}
\only<3>{Move derivative over probability}
\only<4>{Assume softmax form}

\only<5->{
Approximate with empirical estimate for $m$ sample paths from $\pi$
\begin{equation}
\nabla_\theta U(\theta) \approx \frac{1}{m} \sum_1^m \nabla_\theta
\log P(r^i; \theta) R(\tau^i)
\end{equation}
}

\end{frame}

\begin{frame}{Policy Gradient Intuition}

  \gfx{pg-intuition}{.8}

  \begin{itemize}
\item Increase probability of paths	with positive $R$
\item Decrease probability of paths with negagive $R$
  \end{itemize}

\end{frame}

\begin{frame}{Extensions}

  \begin{itemize}
    \item Consider baseline $b$ (e.g., path averaging)
\begin{equation}
\nabla_\theta U(\theta) \approx \frac{1}{m} \sum_1^m \nabla_\theta
\log P(r^i; \theta) (R(\tau^i) - b(\tau))
\end{equation}
\item Combine with value estimation (critic)
  \begin{itemize}
    \item Critic: Updates action-value function parameters
    \item Actor: Updates policy parameters in direction suggested by critic
   \end{itemize}
  \end{itemize}

\end{frame}

\end{document}