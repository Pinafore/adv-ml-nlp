\documentclass[compress]{beamer}



%

\usepackage[T1]{fontenc}
%\usepackage{kmath,kerkis}
%\usepackage{fouriernc}
\usepackage[adobe-utopia]{mathdesign}
%\usepackage{arev}
\usepackage{times}
\usepackage{natbib}

\usepackage[noend]{algpseudocode}
\usepackage{xmpmulti}
\usepackage{dsfont}
\usepackage{amsmath}

\usepackage{graphicx,float,wrapfig, bbm}
\usepackage{amsfonts, comment, bbold}
\usepackage{mdwlist}
\usepackage{subfigure}
\usepackage{colortbl}
\usepackage{mathrsfs}


\usepackage{multirow}




% packages

\usepackage{amsfonts}

% environments

\newenvironment{packed_enumerate}{
  \begin{enumerate}
    \setlength{\topsep}{0pt}
    \setlength{\itemsep}{2pt}
    \setlength{\parskip}{0pt}
    \setlength{\parsep}{0pt}
}{\end{enumerate}}

\newenvironment{stepit}
 {\begin{itemize}[<+-|alert@+>]}
   {\end{itemize}}

% commands

\newcommand{\Norm}[3]{\mathcal{N}\left( #1 \g #2, #3 \right)}
\newcommand{\popshow}[2]{\only<#1->{\alert<#1>{#2}}}
\newcommand{\x}{\mathbf{x}}
\newcommand{\ex}[1]{\mbox{exp}\left\{ #1\right\} }
\newcommand{\e}[2]{\mathbb{E}_{#1}\left[ #2 \right] }
\newcommand{\g}{\, | \,}
\newcommand{\indpt}{\protect\mathpalette{\protect\independenT}{\perp}}
\def\independenT#1#2{\mathrel{\rlap{$#1#2$}\mkern2mu{#1#2}}}
\newcommand{\E}{\textrm{E}}
\newcommand{\R}{\textrm{R}}
\newcommand{\realline}{\mathbb{R}}
\newcommand{\data}{{\cal D}}
\newcommand{\loglik}{{\cal L}}
\newcommand{\grad}[2]{ \frac{\partial{#1}}{\partial#2}}
\newcommand{\dir}[1]{\mbox{Dir}(#1)}
\newcommand{\mult}[1]{\mbox{Mult}( #1)}
\newcommand{\G}[1]{\Gamma \left( \textstyle #1 \right)}
\newcommand{\ind}[1]{\mathds{1}\left[ #1 \right] }
\newcommand{\norm}[1]{\left\lVert#1\right\rVert}

\newcommand{\class}[1]{ \texttt{#1}}
\newcommand{\term}[1]{ ``#1''}
\newcommand{\tcword}[0]{ w }
\newcommand{\docsetlabeled}[0]{ D }
\newcommand{\onedoclabeled}[0]{ d }
\newcommand{\tcposindex}[0]{ i }
\newcommand{\myblue}[1]{ {\textbf #1 }}
\newcommand{\dnrm}[1]{ _{\mbox{\textsc{ #1 }}}}
\newcommand{\argmax}[0]{ \arg \max }
\newcommand{\tcjclass}[0]{c_j}
\newcommand{\maths}[1]{ {\bf #1}}

\newcommand{\fsi}[2]{
\begin{frame}[plain]
\vspace*{-1pt}
\makebox[\linewidth]{\includegraphics[width=\paperwidth]{#1}}
\begin{center}
#2
\end{center}
\end{frame}
}





% complexity
\renewcommand{\O}{\mathcal{O}}



\setbeamersize{text margin left=0.5cm}
\setbeamersize{text margin right=0.5cm}
\setbeamercolor{alert}{fg=red!75!black}


\usetheme{default}
\useinnertheme{circles}
\useoutertheme{split}
\usecolortheme{seahorse}
% \usecolortheme{dove}
% \usecolortheme{seagull}
%\usecolortheme{default}
% \usecolortheme{dolphin}
\usefonttheme{structurebold}
%\usefonttheme{serif}

\setbeamertemplate{navigation symbols}{}
\setbeamertemplate{headline}{}
\setbeamertemplate{footline}{}
\setbeamerfont{itemize/enumerate subbody}{size=\normalsize}
\setbeamerfont{itemize/enumerate subsubbody}{size=\normalsize}
\setbeamercolor{itemize item}{fg=gray}
\setbeamercolor{enumerate item}{fg=gray}
\setbeamercolor{itemize item}{fg=gray}
\setbeamercolor{itemize subitem}{fg=gray}
\setbeamercolor{item projected}{bg=gray}
\setbeamercolor{subitem projected}{bg=gray}

\newcommand{\explain}[2]{\underbrace{#2}_{\mbox{\footnotesize{#1}}}}
\newcommand\hmmax{0}
\newcommand\bmmax{0}

\newenvironment{bullets}
{\begin{itemize} \setlength{\itemsep}{10pt}}
{\end{itemize}}

\newcommand{\mygraphic}[2]{
  \begin{beamercolorbox}[colorsep*=4pt]{black math}
    \begin{center}
      \includegraphics[#1]{#2}
    \end{center}
  \end{beamercolorbox}
}

\setbeamercolor{structure}{bg=gray}
\setbeamercolor{section in head/foot}{bg=gray}
\setbeamercolor{palette primary}{bg=lightgray}


\usepackage{minted}

\usetheme[pageofpages=of,                    % String used between the current page and the
                                             % total page count.
          bullet=circle,                     % Use circles instead of squares for bullets.
          titleline=true,                    % Show a line below the frame title.
          showdate=true,                     % show the date on the title page
          alternativetitlepage=true,         % Use the fancy title page.
          titlepagelogo=culogo,              % Logo for the first page.
          % Logo for the header on first page.
          headerlogo=boulder_cs,
          ]{UCBoulder}

\usecolortheme{ucdblack}
\author{Advanced Machine Learning for NLP}


\institute[Boyd-Graber] % (optional, but mostly needed)
{Jordan Boyd-Graber}


\AtBeginSection[] % "Beamer, do the following at the start of every section"
{ \begin{frame} \frametitle{Outline} % make a frame titled "Outline"
\tableofcontents[currentsection] % show TOC and highlight current section
\end{frame} }



\usepackage{tikz-dependency}
%\usepackage{tikz-qtree}
\usepackage{qtree}
\usepackage{pdfpages}

\newcommand{\gfx}[2]{
\begin{center}
	\includegraphics[width=#2\linewidth]{distsim/#1}
\end{center}
}
\title{Distributional Semantics}
\date{Slides Adapted from Yoav Goldberg and Omer Levy}

\begin{document}

\tikzstyle{every picture}+=[remember picture]



\begin{frame}
  \titlepage
\end{frame}


\begin{frame}{From Distributional to Distributed Semantics}
    \begin{block}{The new kid on the block}
        \begin{itemize}
            \item Deep learning / neural networks
            \item ``Distributed'' word representations
                \begin{itemize}
                    \item Feed text into neural-net. Get back ``word
                        embeddings''.
                    \item Each word is represented as a low-dimensional vector.
                    \item Vectors capture ``semantics''
                \end{itemize}
            \item \texttt{word2vec} (Mikolov et al)
        \end{itemize}
    \end{block}
\end{frame}

\begin{frame}{From Distributional to Distributed Semantics}
    \begin{block}{This part of the talk}
        \begin{itemize}
            \item \texttt{word2vec} as a black box
            \item a peek inside the black box
            \item relation between word-embeddings and the distributional
                representation
            \item tailoring word embeddings to your needs using
                \texttt{word2vec}
        \end{itemize}
    \end{block}
\end{frame}

\begin{frame}{word2vec}

    \includegraphics[width=\textwidth]{distsim/word2vec_site}

\end{frame}

\begin{frame}{word2vec}

    \centering
    \includegraphics[width=0.8\textwidth]{distsim/w2v_flow.png}

\end{frame}

\begin{frame}{word2vec}

    \begin{itemize}
        \item dog
            \begin{itemize}
                \item cat, dogs, dachshund, rabbit, puppy, poodle, rottweiler,
                    mixed-breed, doberman, pig
            \end{itemize}
        \item sheep
            \begin{itemize}
                \item cattle, goats, cows, chickens, sheeps, hogs, donkeys,
                    herds, shorthorn, livestock
            \end{itemize}
        \item november
            \begin{itemize}
                \item october, december, april, june, february, july, september,
                    january, august, march
            \end{itemize}
        \item jerusalem
            \begin{itemize}
                \item tiberias, jaffa, haifa, israel, palestine, nablus, damascus
                    katamon, ramla, safed
            \end{itemize}
        \item teva
            \begin{itemize}
                \item pfizer, schering-plough, novartis, astrazeneca,
                    glaxosmithkline, sanofi-aventis, mylan, sanofi, genzyme, pharmacia
            \end{itemize}
    \end{itemize}
\end{frame}


\begin{frame}{Working with Dense Vectors}

    \begin{block}{Word Similarity}
        \begin{itemize}
            \item Similarity is calculated using \textit{cosine
                similarity}:
        \end{itemize}
        \[sim(\vec{dog}, \vec{cat}) = \frac{\vec{dog} \cdot \vec{cat}}{||\vec{dog}|| \; ||\vec{cat}||}\]
        \begin{itemize}
            \item For normalized vectors ($||x|| = 1$), this is equivalent to a
                dot product:
        \end{itemize}
        \[sim(\vec{dog}, \vec{cat}) = \vec{dog} \cdot \vec{cat}\]
        \begin{itemize}
            \item \textbf{Normalize the vectors when loading them.}
        \end{itemize}
    \end{block}


\end{frame}

\begin{frame}[fragile]{Working with Dense Vectors}
    \begin{block}{Finding the most similar words to $\vec{dog}$}
        \begin{itemize}
            \item Compute the similarity from word $\vec{v}$ to all other words.
                \pause
            \item This is a \textbf{single matrix-vector product}: $W \cdot \vec{v}^\top$
                \begin{center}
                    \includegraphics[width=0.4\textwidth]{distsim/Wv.png}
                \end{center}
                \pause
            \item Result is a $|V|$ sized vector of similarities.
            \item Take the indices of the $k$-highest values.
                \pause
            \item \textbf{FAST! for 180k words, d=300: $\sim$30ms}
        \end{itemize}
    \end{block}
\end{frame}

\begin{frame}[fragile]{Working with Dense Vectors}
    \begin{block}{Most Similar Words, in python+numpy code}
   \begin{minted}{python}
W,words = load_and_norm_vectors("vecs.txt")
# W and words are numpy arrays.
w2i = {w:i for i,w in enumerate(words)}

dog = W[w2i['dog']] # get the dog vector

sims = W.dot(dog)   # compute similarities

most_similar_ids = sims.argsort()[-1:-10:-1]
sim_words = words[most_similar_ids]
   \end{minted}
   \end{block}
\end{frame}

\begin{frame}[fragile]{Working with Dense Vectors}
    \begin{block}{Similarity to a group of words}
        \begin{itemize}
            \item ``Find me words most similar to cat, dog and cow''.
            \item Calculate the pairwise similarities and sum them:
                \[W \cdot \vec{cat} + W \cdot \vec{dog} + W \cdot \vec{cow} \]
            \item Now find the indices of the highest values as before.
                \pause
        \end{itemize}
        \begin{itemize}
            \item Matrix-vector products are wasteful. \textbf{Better option:}
                \[W \cdot (\vec{cat} + \vec{dog} + \vec{cow}) \]
        \end{itemize}
    \end{block}
\end{frame}

\begin{frame}{}

    \centering
    Working with dense word vectors can be very efficient.

    \vspace{2em}

    \pause
    But where do these vectors come from?

\end{frame}

\begin{frame}{How does word2vec work?}

    word2vec implements several different algorithms:

    \begin{block}{Two training methods}
    \begin{itemize}
        \item \textbf<2>{Negative Sampling}
        \item Hierarchical Softmax
    \end{itemize}
    \end{block}
    \begin{block}{Two context representations}
    \begin{itemize}
        \item Continuous Bag of Words (CBOW)
        \item \textbf<2>{Skip-grams}
    \end{itemize}
    \end{block}

    \pause
    \vspace{1em}
    We'll focus on skip-grams with negative sampling

    \vspace{1em}
    intuitions apply for other models as well
\end{frame}

\begin{frame}{How does word2vec work?}
    \begin{itemize}
        \item Represent each word as a $d$ dimensional vector.
        \item Represent each context as a $d$ dimensional vector.
        \item Initalize all vectors to random weights.
        \item Arrange vectors in two matrices, $W$ and $C$.
    \end{itemize}
    \includegraphics[width=0.5\textwidth]{distsim/WC.png}
\end{frame}

\begin{frame}{How does word2vec work?}
    While more text:
    \begin{itemize}
        \item Extract a word window:
    \end{itemize}
    \scalebox{0.8}{
\begin{tabular}{ccccccccc}
\texttt{A springer is} [ & \texttt{a}     & \texttt{cow}   & \texttt{or}
& \texttt{\textbf{heifer}} & \texttt{close} & \texttt{to} &
\texttt{calving} & ] \texttt{.} \\
          & $c_1$ & $c_2$ & $c_3$ & $w$            &  $c_4$ & $c_5$  & $c_6$ &  \\
      \end{tabular}}
      \only<1>{
      \begin{itemize}
          \item $w$ is the focus word vector (row in $W$).
          \item $c_i$ are the context word vectors (rows in $C$).
      \end{itemize}
        }

        \pause
    \begin{itemize}
        \item Try setting the vector values such that:
            \[\sigma(w\cdot~c_1) + \sigma(w\cdot~c_2) + \sigma(w\cdot~c_3) +
            \sigma(w\cdot~c_4) + \sigma(w\cdot~c_5) + \sigma(w\cdot~c_6)\]
            is \textbf{high}
    \end{itemize}
        \pause
    \begin{itemize}
        \item Create a corrupt example by choosing a random word $w'$
        \scalebox{0.8}{
    \begin{tabular}{ccccccccc}
    [ & \texttt{a}     & \texttt{cow}   & \texttt{or}
    & \texttt{\textbf{comet}} & \texttt{close} & \texttt{to} &
    \texttt{calving} & ] \\
              & $c_1$ & $c_2$ & $c_3$ & $w'$            &  $c_4$ & $c_5$  & $c_6$ &  \\
          \end{tabular}}
    \end{itemize}

    \begin{itemize}
        \item Try setting the vector values such that:
            \[\sigma(w'\cdot~c_1) + \sigma(w'\cdot~c_2) + \sigma(w'\cdot~c_3) +
            \sigma(w'\cdot~c_4) + \sigma(w'\cdot~c_5) + \sigma(w'\cdot~c_6)\]
            is \textbf{low}
    \end{itemize}
\end{frame}

\begin{frame}{How does word2vec work?}

    The training procedure results in:
    \begin{itemize}
        \item $w\cdot c$ for \textbf{good} word-context pairs is \textbf{high}
        \item $w\cdot c$ for \textbf{bad} word-context pairs is \textbf{low}
        \item $w\cdot c$ for \textbf{ok-ish} word-context pairs is \textbf{neither high nor low}
    \end{itemize}

    \vspace{1em}
    As a result:
    \begin{itemize}
        \item Words that share many contexts get close to each other.
        \item Contexts that share many words get close to each other.
    \end{itemize}

    \vspace{1em}
    At the end, word2vec throws away $C$ and returns $W$.



\end{frame}

\begin{frame}{Reinterpretation}

    Imagine we didn't throw away $C$. Consider the product $WC^\top$
    \pause
    \vspace{1em}

    \begin{center}
    \includegraphics[width=0.6\textwidth]{distsim/word_context_matrix.png}
    \end{center}

    \vspace{1em}
    The result is a matrix $M$ in which:
    \begin{itemize}
        \item Each row corresponds to a word.
        \item Each column corresponds to a context.
        \item Each cell: $w\cdot c$, association between
            word and context.
    \end{itemize}

\end{frame}

\begin{frame}{Reinterpretation}

    \includegraphics[width=0.5\textwidth]{distsim/word_context_matrix.png}

    \vspace{1em}
    Does this remind you of something?
    \pause

    \vspace{0.3em}
    Very similar to SVD over distributional representation:
    \vspace{1em}

    \includegraphics[width=0.5\textwidth]{distsim/svd.png}

\end{frame}

\begin{frame}{Relation between SVD and word2vec}
    \begin{block}{SVD}
        \begin{itemize}
            \item Begin with a word-context matrix.
            \item Approximate it with a product of low rank (thin) matrices.
            \item Use thin matrix as word representation.
        \end{itemize}
    \end{block}
    \begin{block}{\texttt{word2vec} (skip-grams, negative sampling)}
        \begin{itemize}
            \item Learn thin word and context matrices.
            \item These matrices can be thought of as approximating an implicit word-context
                matrix.
                \begin{itemize}
                    \item Levy and Goldberg (NIPS 2014) show that this
                        implicit matrix is related to the well-known PPMI matrix.
                \end{itemize}
        \end{itemize}
    \end{block}
\end{frame}

\begin{frame}{Relation between SVD and word2vec}
word2vec is a dimensionality reduction technique over an (implicit) word-context
matrix.

\vspace{1em}

Just like SVD.

\vspace{1em}
With few tricks {\footnotesize(Levy, Goldberg and Dagan, in submission)}
we can get SVD to perform just as well as \texttt{word2vec}.

\pause
\vspace{1em}
However, \texttt{word2vec}\ldots

\begin{itemize}
    \item \textbf{\ldots works without building / storing the actual matrix in memory.}
    \item \textbf{\ldots is very fast to train, can use multiple threads.}
    \item \textbf{\ldots can easily scale to huge data and very large word
and context vocabularies.}
\end{itemize}
\end{frame}

%\begin{frame}{word2vec vs. distributional}
%    @@ TODO: remove / move this slide? @@
%    \begin{block}{When should we consider the sparse distributional representation?}
%        \begin{itemize}
%            \item Relatively small amounts of data.
%            \item The input comes as counted word-context pairs.
%            \item Want to interpret the different dimensions.
%            \item Want a more complex similarity measure, i.e. directional
%                similarity.
%        \end{itemize}
%    \end{block}
%\end{frame}

\begin{frame}{}
    \centering
    Beyond word2vec

\end{frame}

\begin{frame}{Beyond word2vec}

    \begin{itemize}
        \item word2vec is factorizing a word-context matrix.
        \item The content of this matrix affects the resulting similarities.
        \item word2vec allows you to specify a \textit{window size}.
        \item But what about other types of contexts?
    \end{itemize}

    \begin{itemize}
        \item Example: \textbf{dependency contexts} {\footnotesize (Levy and
            Dagan, ACL 2014)}
    \end{itemize}

\end{frame}

{
\setbeamercolor{background canvas}{bg=}
\includepdf[pages=-]{distsim/dep_embeddings.pdf}
}

\begin{frame}{Context matters}

    \begin{block}{Choose the correct contexts for your application}
        \begin{itemize}
            \item larger window sizes -- more topical
            \item dependency relations -- more functional
                \pause
            \item only noun-adjective relations
%                \scalebox{0.8}{
%                \begin{tabular}{cc}
%                    & \textbf{all dependencies} \\
%                            \hline
%                                   \textbf{sandwich}:
%                                   & satay, sandwiches, ravioli, doughnut, dessert, \\
%                                   & casserole, steak, shortbread, omelette, risotto \\
%                           \hline
%                           & \textbf{only noun-adj dependencies} \\
%                           \hline
%                                \textbf{sandwich}:
%                                & pizza, sandwiches, pan, pastry, doughnut \\
%                                & biscuit, doughnuts, burger, dessert, pies \\
%                            \end{tabular}}
%                \pause
            \item only verb-subject relations
                \pause
            \item context: time of the current message
            \item context: user who wrote the message
                \pause
            \item \ldots
            \item the sky is the limit
        \end{itemize}
    \end{block}

\end{frame}


\begin{frame}{Software}

    \begin{block}{\texttt{word2vecf}}
        \texttt{https://bitbucket.org/yoavgo/word2vecf}
    \begin{itemize}
        \item Extension of \texttt{word2vec}.
        \item Allows saving the context matrix.
        \item Allows using \textbf{arbitraty contexts}.
            \begin{itemize}
                \item Input is a (large) file of word context pairs.
            \end{itemize}
    \end{itemize}
    \end{block}
\end{frame}

\begin{frame}{Software}

    \begin{block}{\texttt{hyperwords}}
        \texttt{https://bitbucket.org/omerlevy/hyperwords/}
    \begin{itemize}
        \item Python library for working with either sparse or dense word vectors (similarity, analogies).
        \item Scripts for creating dense representations using word2vecf or SVD.
        \item Scripts for creating sparse distributional representations.
    \end{itemize}
    \end{block}

\end{frame}


\begin{frame}{Software}
    \begin{block}{\texttt{dissect}}
        \texttt{http://clic.cimec.unitn.it/composes/toolkit/}
    \begin{itemize}
        \item Given vector representation of words\ldots
        \item \ldots derive vector representation of phrases/sentences
        \item Implements various composition methods
    \end{itemize}
    \end{block}
\end{frame}

\begin{frame}{Summary}
    \begin{block}{Distributional Semantics}
        \begin{itemize}
            \item Words in similar contexts have similar meanings.
            \item Represent a word by the contexts it appears in.
            \item But what is a context?
        \end{itemize}
    \end{block}
    \begin{block}{Neural Models (word2vec)}
        \begin{itemize}
            \item Represent each word as dense, low-dimensional vector.
            \item Same intuitions as in distributional vector-space models.
            \item Efficient to run, scales well, modest memory requirement.
            \item Dense vectors are convenient to work with.
            \item Still helpful to think of the context types.
        \end{itemize}
    \end{block}
    \vspace{-5pt}
    \begin{block}{Software}
        \begin{itemize}
        \item Build your own word representations.
        \end{itemize}
    \end{block}
\end{frame}


\end{document}

%%%% examples
