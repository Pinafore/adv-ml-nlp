
\documentclass[compress]{beamer}
\usefonttheme{professionalfonts}



%

\usepackage[T1]{fontenc}
%\usepackage{kmath,kerkis}
%\usepackage{fouriernc}
\usepackage[adobe-utopia]{mathdesign}
%\usepackage{arev}
\usepackage{times}
\usepackage{natbib}

\usepackage[noend]{algpseudocode}
\usepackage{xmpmulti}
\usepackage{dsfont}
\usepackage{amsmath}

\usepackage{graphicx,float,wrapfig, bbm}
\usepackage{amsfonts, comment, bbold}
\usepackage{mdwlist}
\usepackage{subfigure}
\usepackage{colortbl}
\usepackage{mathrsfs}


\usepackage{multirow}




% packages

\usepackage{amsfonts}

% environments

\newenvironment{packed_enumerate}{
  \begin{enumerate}
    \setlength{\topsep}{0pt}
    \setlength{\itemsep}{2pt}
    \setlength{\parskip}{0pt}
    \setlength{\parsep}{0pt}
}{\end{enumerate}}

\newenvironment{stepit}
 {\begin{itemize}[<+-|alert@+>]}
   {\end{itemize}}

% commands

\newcommand{\Norm}[3]{\mathcal{N}\left( #1, #2, #3 \right)}
\newcommand{\popshow}[2]{\only<#1->{\alert<#1>{#2}}}
\newcommand{\x}{\mathbf{x}}
\newcommand{\ex}[1]{\mbox{exp}\left\{ #1\right\} }
\newcommand{\e}[2]{\mathbb{E}_{#1}\left[ #2 \right] }
\newcommand{\g}{\, | \,}
\newcommand{\indpt}{\protect\mathpalette{\protect\independenT}{\perp}}
\def\independenT#1#2{\mathrel{\rlap{$#1#2$}\mkern2mu{#1#2}}}
\newcommand{\E}{\textrm{E}}
\newcommand{\R}{\textrm{R}}
\newcommand{\realline}{\mathbb{R}}
\newcommand{\data}{{\cal D}}
\newcommand{\loglik}{{\cal L}}
\newcommand{\grad}[2]{ \frac{\partial{#1}}{\partial#2}}
\newcommand{\dir}[1]{\mbox{Dir}(#1)}
\newcommand{\mult}[1]{\mbox{Mult}( #1)}
\newcommand{\G}[1]{\Gamma \left( \textstyle #1 \right)}
\newcommand{\ind}[1]{\mathds{1}\left[ #1 \right] }
\newcommand{\norm}[1]{\left\lVert#1\right\rVert}

\newcommand{\class}[1]{ \texttt{#1}}
\newcommand{\term}[1]{ ``#1''}
\newcommand{\tcword}[0]{ w }
\newcommand{\docsetlabeled}[0]{ D }
\newcommand{\onedoclabeled}[0]{ d }
\newcommand{\tcposindex}[0]{ i }
\newcommand{\myblue}[1]{ {\textbf #1 }}
\newcommand{\dnrm}[1]{ _{\mbox{\textsc{ #1 }}}}
\newcommand{\argmax}[0]{ \arg \max }
\newcommand{\tcjclass}[0]{c_j}
\newcommand{\maths}[1]{ {\bf #1}}




% complexity
\renewcommand{\O}{\mathcal{O}}



\setbeamersize{text margin left=0.5cm}
\setbeamersize{text margin right=0.5cm}
\setbeamercolor{alert}{fg=red!75!black}

\usetheme{default}
\useinnertheme{circles}
\useoutertheme{split}
\usecolortheme{seahorse}
% \usecolortheme{dove}
% \usecolortheme{seagull}
%\usecolortheme{default}
% \usecolortheme{dolphin}
\usefonttheme{structurebold}
%\usefonttheme{serif}

\setbeamertemplate{navigation symbols}{}
\setbeamertemplate{headline}{}
\setbeamertemplate{footline}{}
\setbeamerfont{itemize/enumerate subbody}{size=\normalsize}
\setbeamerfont{itemize/enumerate subsubbody}{size=\normalsize}
\setbeamercolor{itemize item}{fg=gray}
\setbeamercolor{enumerate item}{fg=gray}
\setbeamercolor{itemize item}{fg=gray}
\setbeamercolor{itemize subitem}{fg=gray}
\setbeamercolor{item projected}{bg=gray}
\setbeamercolor{subitem projected}{bg=gray}


\newenvironment{bullets}
{\begin{itemize} \setlength{\itemsep}{10pt}}
{\end{itemize}}

\newcommand{\mygraphic}[2]{
  \begin{beamercolorbox}[colorsep*=4pt]{black math}
    \begin{center}
      \includegraphics[#1]{#2}
    \end{center}
  \end{beamercolorbox}
}

\setbeamercolor{structure}{bg=gray}
\setbeamercolor{section in head/foot}{bg=gray}
\setbeamercolor{palette primary}{bg=lightgray}


\usepackage{minted}

\usetheme[pageofpages=of,                    % String used between the current page and the
                                             % total page count.
          bullet=circle,                     % Use circles instead of squares for bullets.
          titleline=true,                    % Show a line below the frame title.
          showdate=true,                     % show the date on the title page
          alternativetitlepage=true,         % Use the fancy title page.
          titlepagelogo=../../common/culogo,              % Logo for the first page.
          % Logo for the header on first page.
          headerlogo=../../common/boulder_cs,
          ]{UCBoulder}

\usecolortheme{ucdblack}
\author{Introduction to Data Science Algorithms}


\institute[Boyd-Graber and Paul] % (optional, but mostly needed)
{Jordan Boyd-Graber and Michael Paul}


\AtBeginSection[] % "Beamer, do the following at the start of every section"
{ \begin{frame} \frametitle{Outline} % make a frame titled "Outline"
\tableofcontents[currentsection] % show TOC and highlight current section
\end{frame} }

\newcommand{\gfx}[2]{
\begin{center}
	\includegraphics[width=#2\linewidth]{lm/#1}
\end{center}
}
\title{Language Models}
\date{Foundations}

\begin{document}


\frame{\titlepage
}



\begin{frame}{Roadmap}
After this class, you'll be able to:
  \begin{itemize}
    \item Give examples of where we need language models
    \item Evaluate language models
    \item Connection between Bayesian nonparametrics and backoff
  \end{itemize}
\end{frame}


\begin{frame}{Language models}
\begin{itemize}
\item {\bf Language models} answer the question: {\em How likely is a string of English words good English?}
\item Autocomplete on phones and websearch
\item Creating English-looking documents
\item Very common in machine translation systems
\begin{itemize}
\item Help with reordering / style
\begin{equation*}
p_{\mbox{\sc lm}}(\mbox{the house is small}) > p_{\mbox{\sc lm}}(\mbox{small the is house})
\end{equation*}
\item Help with word choice
\begin{equation*}
p_\text{\sc lm}(\mbox{I am going home}) > p_\text{\sc lm}(\mbox{I am going house})
\end{equation*}
\end{itemize}
\end{itemize}

\end{frame}

\begin{frame}{Why language models?}

  \begin{itemize}
    \item Like sorting for computer science
    \item Language models essential for many NLP applications
    \item Optimized for performance and runtime
  \end{itemize}

\end{frame}


\begin{frame}{N-Gram Language Models}
\begin{itemize}
\item Given: a string of English words \maths{$W = w_1,w_2,w_3,...,w_n$}
\item Question: what is \maths{$p(W)$}?
\item Sparse data: Many good English sentences will not have been seen before
\item[$\rightarrow$] Decomposing \maths{$p(W)$} using the chain rule:
\begin{align*}
p(w_1,w_2,w_3,...,w_n) = & \\ p(w_1)\;p(w_2|w_1)&\;p(w_3|w_1,w_2) \dots p(w_n|w_1,w_2,...w_{n-1})
\end{align*}
(not much gained yet, \maths{$p(w_n|w_1,w_2,...w_{n-1})$} is equally sparse)
\end{itemize}

\end{frame}

\begin{frame}{Markov Chain}
\begin{itemize}
\item {\bf Markov independence assumption}:
\begin{itemize}
\item only previous history matters
\item limited memory: only last $k$ words are included in history \\(older words less relevant)
\item[$\rightarrow$] {\bf $k$th order Markov model}
\end{itemize}
\item For instance 2-gram language model:
\begin{equation*}
p(w_1,w_2,w_3,...,w_n) \simeq p(w_1)\;p(w_2|w_1)\;p(w_3|w_2) ... p(w_n|w_{n-1})
\end{equation*}
\item What is conditioned on, here \maths{$w_{i-1}$} is called the {\bf history}
\end{itemize}

\end{frame}


\begin{frame}{How good is the LM?}
\begin{itemize}
\item A good model assigns a text of real English \maths{$W$} a high probability
\item This can be also measured with {\bf perplexity}
\begin{align*}
\text{perplexity}(W) & = P(w_1, \dots w_N)^{-\frac{1}{N}} \\
	& = \sqrt[N]{\prod_i^N \frac{1}{P(w_i | w_1 \dots w_{i-1})}}
\end{align*}
\end{itemize}

\end{frame}


%%%%%%%%%%%%%%%%%%%%%%%%%%%%%%%%%%%%%%%%%%%%%%%%%%%%%%%%%%%%%%%%%%%%%%%%%%%%

\begin{frame}{Comparison 1--4-Gram}
\begin{center}
\begin{tabular}{c|r|r|r|r}
word & unigram & bigram & trigram & 4-gram \\ \hline \hline
{\bf i}           & 6.684 & 3.197 & 3.197 & 3.197\\ \hline
{\bf would}       & 8.342 & 2.884 & 2.791 & 2.791\\ \hline
{\bf like}        & 9.129 & 2.026 & 1.031 & 1.290\\ \hline
{\bf to}          & 5.081 & 0.402 & 0.144 & 0.113\\ \hline
{\bf commend}     &15.487 &12.335 & 8.794 & 8.633\\ \hline
{\bf the}         & 3.885 & 1.402 & 1.084 & 0.880\\ \hline
{\bf reporter }  &10.840 & 7.319 & 2.763 & 2.350\\ \hline
{\bf .}           & 4.896 & 3.020 & 1.785 & 1.510\\ \hline
{\bf \textless /s\textgreater}  & 4.828 & 0.005 & 0.000 & 0.000\\ \hline \hline
average          &   &  &   &  \\ \hline
perplexity       & 265.136 & 16.817 & 6.206 & 4.758\\
\end{tabular}
\end{center}

\end{frame}



\begin{frame}{Example: 3-Gram}

\begin{itemize}
\item Counts for trigrams and estimated word probabilities
\begin{center}

\begin{tabular}{c|c|c}
\multicolumn{3}{c}{{\bf the red} (total: 225)}\\[1mm]
word & c. & prob. \\ \hline \hline
{\bf cross} & 123 & 0.547 \\ \hline
{\bf tape} & 31 & 0.138 \\ \hline
{\bf army} & 9 & 0.040 \\ \hline
{\bf card} & 7 & 0.031 \\ \hline
{\bf ,} & 5 & 0.022 \\ \hline
\end{tabular}

\end{center}

\begin{itemize}
\item 225 trigrams in the Europarl corpus start with {\bf the red}
\item 123 of them end with {\bf cross}
\item[$\rightarrow$] maximum likelihood probability is $\frac{123}{225}=0.547$.
\end{itemize}
\pause
\item Can't use ML estimate
\end{itemize}

\end{frame}


\begin{frame}
\frametitle{How do we estimate a probability?}

\begin{itemize}
   \item Assuming a {\bf sparse Dirichlet} prior, \maths{$\alpha<1$} to each count
     \begin{equation}
       \theta_{i} = \frac{n_i + \alpha_i}{\sum_k {n_k + \alpha_k}}
     \end{equation}
   \item $\alpha_i$ is called a smoothing factor, a pseudocount, etc.
     \pause
   \item When $\alpha_i = 1$ for all $i$, it's called ``Laplace smoothing''
     \pause
   \item What is a good value for  \maths{$\alpha$}?
   \item Could be optimized on held-out set to find the ``best'' language model
\end{itemize}
\end{frame}

\begin{frame}{Example: 2-Grams in Europarl}

\begin{center}
\begin{tabular}{|c|c|c|c|} \hline
\bf Count & \multicolumn{2}{c|}{\bf Adjusted count} & \bf Test count \\ \hline
\maths{$c$} & \maths{$(c+1)$} & \maths{$(c+\alpha)$} & \maths{$t_c$}\\ \hline
  0 &  0.00378 &  0.00016 & 0.00016\\ \hline
  1 &  0.00755 &  0.95725 & 0.46235\\ \hline
  2 &  0.01133 &  1.91433 & 1.39946\\ \hline
  3 &  0.01511 &  2.87141 & 2.34307\\ \hline
  4 &  0.01888 &  3.82850 & 3.35202\\ \hline
  5 &  0.02266 &  4.78558 & 4.35234\\ \hline
  6 &  0.02644 &  5.74266 & 5.33762\\ \hline
  8 &  0.03399 &  7.65683 & 7.15074\\ \hline
 10 &  0.04155 &  9.57100 & 9.11927\\ \hline
 20 &  0.07931 & 19.14183 & 18.95948\\ \hline
\end{tabular}
\end{center}
\begin{itemize}
\item Add-{$\alpha$} smoothing with \maths{$\alpha=0.00017$}
\item \maths{$t_c$} are average counts of n-grams in test set that occurred \maths{$c$} times in corpus
\end{itemize}

\pause
\vspace{-5cm}

\begin{block}{Can we do better?}
  In higher-order models, we can learn from similar contexts!
\end{block}

\end{frame}

\begin{frame}{Back-Off}

\begin{itemize}
\item In given corpus, we may never observe
\begin{itemize}
\item {\bf Scottish beer drinkers}
\item {\bf Scottish beer eaters}
\end{itemize}
\item Both have count 0\\[2mm]
$\rightarrow$ our smoothing methods will assign them same probability
\item Better: backoff to bigrams:
\begin{itemize}
\item {\bf beer drinkers}
\item {\bf beer eaters}
\end{itemize}
\end{itemize}


\end{frame}

%%%%%%%%%%%%%%%%%%%%%%%%%%%%%%%%%%%%%%%%%%%%%%%%%%%%%%%%%%%%%%%%%%%%%%%%%%%%

\begin{frame}{Interpolation}
\begin{itemize}
\item Higher and lower order n-gram models have different strengths and weaknesses
\begin{itemize}
\item high-order n-grams are sensitive to more context, but have sparse counts
\item low-order n-grams consider only very limited context, but have robust counts
\end{itemize}
\item Combine them
\begin{equation*}
\begin{split}
p_I(w_3|w_1,w_2) = \; & \phantom{\times \;} \lambda_1 \; p_1(w_3) \\
& + \lambda_2 \; p_2(w_3|w_2) \\
& + \lambda_3 \; p_3(w_3|w_1,w_2)
\end{split}
\end{equation*}
\end{itemize}

\end{frame}

%%%%%%%%%%%%%%%%%%%%%%%%%%%%%%%%%%%%%%%%%%%%%%%%%%%%%%%%%%%%%%%%%%%%%%%%%%%%

\begin{frame}{Back-Off}

\begin{itemize}
\item Trust the highest order language model that contains n-gram
\begin{equation*}
\begin{split}
p^{BO}_n&(w_i|w_{i-n+1},...,w_{i-1}) =\\
& =
\begin{cases}
\alpha_n(w_i|w_{i-n+1},...,w_{i-1}) \\
\phantom{--- ---} \text{if count$_n(w_{i-n+1},...,w_i)>0$} \\[2mm]
d_n(w_{i-n+1},...,w_{i-1}) \; p^{BO}_{n-1}(w_i|w_{i-n+2},...,w_{i-1}) \\
\phantom{--- ---}\text{else}
\end{cases}
\end{split}
\end{equation*}

\item Requires
\begin{itemize}
\item adjusted prediction model \maths{$\alpha_n(w_i|w_{i-n+1},...,w_{i-1})$}
\item discounting function \maths{$d_n(w_1,...,w_{n-1})$}
\end{itemize}
\end{itemize}

\end{frame}
%%%%%%%%%%%%%%%%%%%%%%%%%%%%%%%%%%%%%%%%%%%%%%%%%%%%%%%%%%%%%%%%%%%%%%%%%%%%

\begin{frame}{What's a word?}

\begin{itemize}
	\item There are an infinite number of words
		\begin{itemize}
			\item Possible to develop generative story of how new words are created
			\item Bayesian non-parametrics
		\end{itemize}
	\pause
	\item Defining a vocabulary (the event space)
	\item But how do you handle words outside of your vocabulary?
	\pause
	\begin{itemize}
		\item Ignore? You could win just by ignoring everything
		\item Standard: replace with \texttt{<UNK>} token
	\end{itemize}
	\item Next week: word representations!
\end{itemize}


\end{frame}



%%%%%%%%%%%%%%%%%%%%%%%%%%%%%%%%%%%%%%%%%%%%%%%%%%%%%%%%%%%%%%%%%%%%%%%%%%%%

\begin{frame}{Reducing Vocabulary Size}
\begin{itemize}
\item For instance: each number is treated as a separate token
\item Replace them with a number token {\bf \sc num}
\begin{itemize}
\item but: we want our language model to prefer

\begin{equation*}
p_\text{\sc lm}(\mbox{I pay 950.00 in May 2007}) > p_\text{\sc lm}(\mbox{I pay 2007 in May 950.00})
\end{equation*}

\item not possible with number token

\begin{equation*}
p_\text{\sc lm}(\mbox{I pay {\sc num} in May {\sc num}}) = p_\text{\sc lm}(\mbox{I pay {\sc num} in May {\sc num}})
\end{equation*}

\end{itemize}
\item Replace each digit (with unique symbol, e.g., {\bf @} or {\bf 5}), retain some distinctions

\begin{equation*}
p_\text{\sc lm}(\mbox{I pay 555.55 in May 5555}) > p_\text{\sc lm}(\mbox{I pay 5555 in May 555.55})
\end{equation*}
\end{itemize}

\end{frame}



%%%%%%%%%%%%%%%%%%%%%%%%%%%%%%%%%%%%%%%%%%%%%%%%%%%%%%%%%%%%%%%%%%%%%%%%%%%%


\end{document}
