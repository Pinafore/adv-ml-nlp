
\documentclass[compress]{beamer}
\usefonttheme{professionalfonts}



%

\usepackage[T1]{fontenc}
%\usepackage{kmath,kerkis}
%\usepackage{fouriernc}
\usepackage[adobe-utopia]{mathdesign}
%\usepackage{arev}
\usepackage{times}
\usepackage{natbib}

\usepackage[noend]{algpseudocode}
\usepackage{xmpmulti}
\usepackage{dsfont}
\usepackage{amsmath}

\usepackage{graphicx,float,wrapfig, bbm}
\usepackage{amsfonts, comment, bbold}
\usepackage{mdwlist}
\usepackage{subfigure}
\usepackage{colortbl}
\usepackage{mathrsfs}


\usepackage{multirow}




% packages

\usepackage{amsfonts}

% environments

\newenvironment{packed_enumerate}{
  \begin{enumerate}
    \setlength{\topsep}{0pt}
    \setlength{\itemsep}{2pt}
    \setlength{\parskip}{0pt}
    \setlength{\parsep}{0pt}
}{\end{enumerate}}

\newenvironment{stepit}
 {\begin{itemize}[<+-|alert@+>]}
   {\end{itemize}}

% commands

\newcommand{\Norm}[3]{\mathcal{N}\left( #1, #2, #3 \right)}
\newcommand{\popshow}[2]{\only<#1->{\alert<#1>{#2}}}
\newcommand{\x}{\mathbf{x}}
\newcommand{\ex}[1]{\mbox{exp}\left\{ #1\right\} }
\newcommand{\e}[2]{\mathbb{E}_{#1}\left[ #2 \right] }
\newcommand{\g}{\, | \,}
\newcommand{\indpt}{\protect\mathpalette{\protect\independenT}{\perp}}
\def\independenT#1#2{\mathrel{\rlap{$#1#2$}\mkern2mu{#1#2}}}
\newcommand{\E}{\textrm{E}}
\newcommand{\R}{\textrm{R}}
\newcommand{\realline}{\mathbb{R}}
\newcommand{\data}{{\cal D}}
\newcommand{\loglik}{{\cal L}}
\newcommand{\grad}[2]{ \frac{\partial{#1}}{\partial#2}}
\newcommand{\dir}[1]{\mbox{Dir}(#1)}
\newcommand{\mult}[1]{\mbox{Mult}( #1)}
\newcommand{\G}[1]{\Gamma \left( \textstyle #1 \right)}
\newcommand{\ind}[1]{\mathds{1}\left[ #1 \right] }
\newcommand{\norm}[1]{\left\lVert#1\right\rVert}

\newcommand{\class}[1]{ \texttt{#1}}
\newcommand{\term}[1]{ ``#1''}
\newcommand{\tcword}[0]{ w }
\newcommand{\docsetlabeled}[0]{ D }
\newcommand{\onedoclabeled}[0]{ d }
\newcommand{\tcposindex}[0]{ i }
\newcommand{\myblue}[1]{ {\textbf #1 }}
\newcommand{\dnrm}[1]{ _{\mbox{\textsc{ #1 }}}}
\newcommand{\argmax}[0]{ \arg \max }
\newcommand{\tcjclass}[0]{c_j}
\newcommand{\maths}[1]{ {\bf #1}}




% complexity
\renewcommand{\O}{\mathcal{O}}



\setbeamersize{text margin left=0.5cm}
\setbeamersize{text margin right=0.5cm}
\setbeamercolor{alert}{fg=red!75!black}

\usetheme{default}
\useinnertheme{circles}
\useoutertheme{split}
\usecolortheme{seahorse}
% \usecolortheme{dove}
% \usecolortheme{seagull}
%\usecolortheme{default}
% \usecolortheme{dolphin}
\usefonttheme{structurebold}
%\usefonttheme{serif}

\setbeamertemplate{navigation symbols}{}
\setbeamertemplate{headline}{}
\setbeamertemplate{footline}{}
\setbeamerfont{itemize/enumerate subbody}{size=\normalsize}
\setbeamerfont{itemize/enumerate subsubbody}{size=\normalsize}
\setbeamercolor{itemize item}{fg=gray}
\setbeamercolor{enumerate item}{fg=gray}
\setbeamercolor{itemize item}{fg=gray}
\setbeamercolor{itemize subitem}{fg=gray}
\setbeamercolor{item projected}{bg=gray}
\setbeamercolor{subitem projected}{bg=gray}


\newenvironment{bullets}
{\begin{itemize} \setlength{\itemsep}{10pt}}
{\end{itemize}}

\newcommand{\mygraphic}[2]{
  \begin{beamercolorbox}[colorsep*=4pt]{black math}
    \begin{center}
      \includegraphics[#1]{#2}
    \end{center}
  \end{beamercolorbox}
}

\setbeamercolor{structure}{bg=gray}
\setbeamercolor{section in head/foot}{bg=gray}
\setbeamercolor{palette primary}{bg=lightgray}


\usepackage{minted}

\usetheme[pageofpages=of,                    % String used between the current page and the
                                             % total page count.
          bullet=circle,                     % Use circles instead of squares for bullets.
          titleline=true,                    % Show a line below the frame title.
          showdate=true,                     % show the date on the title page
          alternativetitlepage=true,         % Use the fancy title page.
          titlepagelogo=../../common/culogo,              % Logo for the first page.
          % Logo for the header on first page.
          headerlogo=../../common/boulder_cs,
          ]{UCBoulder}

\usecolortheme{ucdblack}
\author{Introduction to Data Science Algorithms}


\institute[Boyd-Graber and Paul] % (optional, but mostly needed)
{Jordan Boyd-Graber and Michael Paul}


\AtBeginSection[] % "Beamer, do the following at the start of every section"
{ \begin{frame} \frametitle{Outline} % make a frame titled "Outline"
\tableofcontents[currentsection] % show TOC and highlight current section
\end{frame} }

\newcommand{\gfx}[2]{
\begin{center}
	\includegraphics[width=#2\linewidth]{frameworks/#1}
\end{center}
}
\title{Frameworks}
\date{Introduction}

\begin{document}


\frame{\titlepage
Slides adapted from Chris Dyer, Yoav Goldberg, Graham Neubig
}



\begin{frame}{Neural Nets and Language}

\begin{columns}
  \column{.5\linewidth}
  \begin{block}{Language}
    Discrete, structured (graphs, trees)
  \end{block}
  \column{.5\linewidth}
  \begin{block}{Neural-Nets}
    Continuous: poor native support for structure
  \end{block}
\end{columns}

Big challenge: writing code that translates between the \{discrete-structured, continuous\} regimes

\end{frame}


\begin{frame}{Outline}
\begin{itemize}
\item Computation graphs (general)
\item Neural Nets in DyNet
\item RNNs
\item Minibatching
\item New functions
\item Tagging with BiLSTM
\item Structured perceptron
\end{itemize}

\end{frame}

\begin{frame}{Computation Graphs}

\begin{block}{Expression}
\only<1>{  $\vec x$}
\only<2>{ $ \vec x^{\top}$}
\only<3>{$ \vec x^{\top} A$}
\only<4-5>{$ \vec x^{\top} A x$}
\only<6-7>{$\only<7>{\alert<7>{y=}} \vec x^{\top} A x + b \cdot \vec x + c$}
\end{block}

\only<1>{\gfx{cg1}{.3}}
\only<2>{\gfx{cg2}{.5}}
\only<3>{\gfx{cg3}{.4}}
\only<4>{\gfx{cg4}{.4}}
\only<5>{\gfx{cg5}{.7}}
\only<6>{\gfx{cg6}{.7}}
\only<7>{\gfx{cg7}{.6}}

\only<2>{
\begin{itemize}
  \item Edge: function argument / data dependency
  \item A node with an incoming edge is a function $F \equiv f(u)$ edge's tail
    node
\item A node computes its value and the value of its derivative w.r.t each argument (edge) times a derivative $\grad{f}{u}$
\end{itemize}
}

\only<3>{Functions can be nullary, unary, binary, \dots n-ary. Often they are unary or binary.}

\only<4>{Computation graphs are (usually) directed and acyclic}

\only<7>{Variable names label nodes}

\end{frame}


\begin{frame}{Algorithms}

\begin{itemize}
\item Graph construction
\item Forward propagation
\begin{itemize}
\item Loop over nodes in topological order
\item Compute the value of the node given its inputs
\item Given my inputs, make a prediction (or compute an ``error'' with
  respect to a ``target output'')
\end{itemize}
\item Backward propagation
\begin{itemize}
\item Loop over the nodes in reverse topological order starting with a final goal node
\item Compute derivatives of final goal node value with respect to each edge’s tail node
\item How does the output change if I make a small change to the inputs?
\end{itemize}
\end{itemize}

\end{frame}

\begin{frame}{Forward Propagation}
  \only<1>{\gfx{fp1}{.7}}
  \only<2>{\gfx{fp2}{.7}}
  \only<3>{\gfx{fp3}{.7}}
  \only<4>{\gfx{fp4}{.7}}
  \only<5>{\gfx{fp5}{.7}}
  \only<6>{\gfx{fp6}{.7}}
  \only<7>{\gfx{fp7}{.7}}
  \only<8>{\gfx{fp8}{.7}}
\end{frame}

\begin{frame}{Constructing Graphs}

\begin{columns}
\column{.5\linewidth}
\begin{block}{Static declaration}
  \begin{itemize}
    \item Define architecture, run data through
    \item PROS: Optimization, hardware support
    \item CONS: Structured data ugly, graph language
  \end{itemize}
\end{block}
Torch, Theano, Tensorflow
\column{.5\linewidth}
\begin{block}{Dynamic declaration}
  \begin{itemize}
  \item Graph implicit with data
  \item PROS: Native language, interleave construction/evaluation
  \item CONS: Slower, computation can be wasted
  \end{itemize}
\end{block}
Stan, Chainer, \alert<2>{DyNet}
\end{columns}
\end{frame}


\begin{frame}{Dynamic Hierarchy in Language}

\begin{itemize}
  \item Language is hierarchical
    \only<3->{
    \begin{itemize}
      \item Graph should reflect this reality
        \item Traditional flow-control best for processing
          \end{itemize}
\item Combinatorial algorithms (e.g., dynamic programming)
\item Exploit independencies to compute over a large space of
  operations tractably
}
\end{itemize}
  \vspace{-.6cm}
\only<2>{
  \gfx{hierarchy}{.9}
}
\end{frame}

\begin{frame}{DyNet}
\begin{itemize}
\item Before DyNet:
  \begin{itemize}
    \item AD libraries are fast and good, lack deep learning must-haves (GPUs, optimization algorithms, primitives for implementing RNNs, etc.)
    \item Deep learning toolkits don't support dynamic graphs well
    \end{itemize}
\item DyNet is a hybrid between a generic autodiff library and a Deep
  learning toolkit
\begin{itemize}
  \item It has the flexibility of a good AD library
    \item It has most obligatory DL primitives\footnote{Although the
        emphasis is dynamic operation, it can run perfectly well in
        ``static mode''. It's quite fast too! But if you're happy with
        that, probably stick to TensorFlow/Theano/Torch.}
      \item Useful for RL over structure (need this later)
\end{itemize}
\end{itemize}
\end{frame}

\begin{frame}{DyNet}

\begin{itemize}
\item C++ backend based on Eigen (like TensorFlow)
\item Custom (``quirky'') memory management
\item A few well-hidden assumptions make the graph construction and execution very fast.
\item Thin Python wrapper on C++ API
\end{itemize}

\end{frame}

\end{document}