

\documentclass[compress]{beamer}
\usefonttheme{professionalfonts}



%

\usepackage[T1]{fontenc}
%\usepackage{kmath,kerkis}
%\usepackage{fouriernc}
\usepackage[adobe-utopia]{mathdesign}
%\usepackage{arev}
\usepackage{times}
\usepackage{natbib}

\usepackage[noend]{algpseudocode}
\usepackage{xmpmulti}
\usepackage{dsfont}
\usepackage{amsmath}

\usepackage{graphicx,float,wrapfig, bbm}
\usepackage{amsfonts, comment, bbold}
\usepackage{mdwlist}
\usepackage{subfigure}
\usepackage{colortbl}
\usepackage{mathrsfs}


\usepackage{multirow}




% packages

\usepackage{amsfonts}

% environments

\newenvironment{packed_enumerate}{
  \begin{enumerate}
    \setlength{\topsep}{0pt}
    \setlength{\itemsep}{2pt}
    \setlength{\parskip}{0pt}
    \setlength{\parsep}{0pt}
}{\end{enumerate}}

\newenvironment{stepit}
 {\begin{itemize}[<+-|alert@+>]}
   {\end{itemize}}

% commands

\newcommand{\Norm}[3]{\mathcal{N}\left( #1, #2, #3 \right)}
\newcommand{\popshow}[2]{\only<#1->{\alert<#1>{#2}}}
\newcommand{\x}{\mathbf{x}}
\newcommand{\ex}[1]{\mbox{exp}\left\{ #1\right\} }
\newcommand{\e}[2]{\mathbb{E}_{#1}\left[ #2 \right] }
\newcommand{\g}{\, | \,}
\newcommand{\indpt}{\protect\mathpalette{\protect\independenT}{\perp}}
\def\independenT#1#2{\mathrel{\rlap{$#1#2$}\mkern2mu{#1#2}}}
\newcommand{\E}{\textrm{E}}
\newcommand{\R}{\textrm{R}}
\newcommand{\realline}{\mathbb{R}}
\newcommand{\data}{{\cal D}}
\newcommand{\loglik}{{\cal L}}
\newcommand{\grad}[2]{ \frac{\partial{#1}}{\partial#2}}
\newcommand{\dir}[1]{\mbox{Dir}(#1)}
\newcommand{\mult}[1]{\mbox{Mult}( #1)}
\newcommand{\G}[1]{\Gamma \left( \textstyle #1 \right)}
\newcommand{\ind}[1]{\mathds{1}\left[ #1 \right] }
\newcommand{\norm}[1]{\left\lVert#1\right\rVert}

\newcommand{\class}[1]{ \texttt{#1}}
\newcommand{\term}[1]{ ``#1''}
\newcommand{\tcword}[0]{ w }
\newcommand{\docsetlabeled}[0]{ D }
\newcommand{\onedoclabeled}[0]{ d }
\newcommand{\tcposindex}[0]{ i }
\newcommand{\myblue}[1]{ {\textbf #1 }}
\newcommand{\dnrm}[1]{ _{\mbox{\textsc{ #1 }}}}
\newcommand{\argmax}[0]{ \arg \max }
\newcommand{\tcjclass}[0]{c_j}
\newcommand{\maths}[1]{ {\bf #1}}




% complexity
\renewcommand{\O}{\mathcal{O}}



\setbeamersize{text margin left=0.5cm}
\setbeamersize{text margin right=0.5cm}
\setbeamercolor{alert}{fg=red!75!black}

\usetheme{default}
\useinnertheme{circles}
\useoutertheme{split}
\usecolortheme{seahorse}
% \usecolortheme{dove}
% \usecolortheme{seagull}
%\usecolortheme{default}
% \usecolortheme{dolphin}
\usefonttheme{structurebold}
%\usefonttheme{serif}

\setbeamertemplate{navigation symbols}{}
\setbeamertemplate{headline}{}
\setbeamertemplate{footline}{}
\setbeamerfont{itemize/enumerate subbody}{size=\normalsize}
\setbeamerfont{itemize/enumerate subsubbody}{size=\normalsize}
\setbeamercolor{itemize item}{fg=gray}
\setbeamercolor{enumerate item}{fg=gray}
\setbeamercolor{itemize item}{fg=gray}
\setbeamercolor{itemize subitem}{fg=gray}
\setbeamercolor{item projected}{bg=gray}
\setbeamercolor{subitem projected}{bg=gray}


\newenvironment{bullets}
{\begin{itemize} \setlength{\itemsep}{10pt}}
{\end{itemize}}

\newcommand{\mygraphic}[2]{
  \begin{beamercolorbox}[colorsep*=4pt]{black math}
    \begin{center}
      \includegraphics[#1]{#2}
    \end{center}
  \end{beamercolorbox}
}

\setbeamercolor{structure}{bg=gray}
\setbeamercolor{section in head/foot}{bg=gray}
\setbeamercolor{palette primary}{bg=lightgray}


\usepackage{minted}

\usetheme[pageofpages=of,                    % String used between the current page and the
                                             % total page count.
          bullet=circle,                     % Use circles instead of squares for bullets.
          titleline=true,                    % Show a line below the frame title.
          showdate=true,                     % show the date on the title page
          alternativetitlepage=true,         % Use the fancy title page.
          titlepagelogo=../../common/culogo,              % Logo for the first page.
          % Logo for the header on first page.
          headerlogo=../../common/boulder_cs,
          ]{UCBoulder}

\usecolortheme{ucdblack}
\author{Introduction to Data Science Algorithms}


\institute[Boyd-Graber and Paul] % (optional, but mostly needed)
{Jordan Boyd-Graber and Michael Paul}


\AtBeginSection[] % "Beamer, do the following at the start of every section"
{ \begin{frame} \frametitle{Outline} % make a frame titled "Outline"
\tableofcontents[currentsection] % show TOC and highlight current section
\end{frame} }

\newcommand{\gfx}[2]{
\begin{center}
	\includegraphics[width=#2\linewidth]{frameworks/#1}
\end{center}
}
\title{Frameworks}
\date{Recurrent Neural Networks in DyNet}

\begin{document}


\frame{\titlepage
Slides adapted from Chris Dyer, Yoav Goldberg, Graham Neubig
}

\begin{frame}{Recurrent Neural Networks}

\begin{itemize}
\item NLP is full of sequential data
\begin{itemize}
\item Words in sentences
\item Characters in words
\item Sentences in discourse
\end{itemize}
\pause
\item How do we represent an arbitrarily long history?
\pause
we will train neural networks to build a representation of these arbitrarily big sequences
\end{itemize}

\end{frame}

\begin{frame}{Recurrent}

  \gfx{ff_vs_rnn}{.9}

\end{frame}

\begin{frame}{Recurrent NN}

  \only<1>{ \gfx{rnn1}{.9}
    How do we train the parameters?
}

  \only<2>{ \gfx{rnn2}{.9}}
  \only<3-4>{ \gfx{rnn3}{.9} Parameter tying}
  \only<4>{
    \vspace{-5cm}
    \begin{block}{Unrolling}
\begin{itemize}
\item Well-formed (DAG) computation graph---we can run backprop
\item Parameters are tied across time, derivatives are aggregated across all time steps
\item ``backpropagation through time''
\end{itemize}
    \end{block}
}
  \only<5>{ \gfx{rnn4}{.9} Each word contributes to gradient}
  \only<6>{ \gfx{rnn5}{.9} Summarize sentence into downstream vector}
  \only<7>{ \gfx{rnn6}{.9} Let's get more concrete: RNN language model}
  \only<8>{ \gfx{rnn7}{.9} }
  \only<9>{ \gfx{rnn8}{.9} }
  \only<10>{ \gfx{rnn9}{.9} Training (log loss from each word)}
\end{frame}


\begin{frame}[fragile]{RNNs in DyNet}

\begin{itemize}
\item Based on ``Builder'' class (for variety of models)
  \item Can also roll your own
    \item Add parameters to model (once)
\begin{minted}[fontsize=\footnotesize]{python}
# RNN (layers=1, input=64, hidden=128, model)
RNN = dy.SimpleRNNBuilder(1, 64, 128, model)
\end{minted}
      \item Add parameters to CG and get initial state (per sentence)
\begin{minted}[fontsize=\footnotesize]{python}
s = RNN.initial_state()
\end{minted}
        \item Update state and access (per input word/character)
\begin{minted}[fontsize=\footnotesize]{python}
s = s.add_input(x_t)
h_t = s.output()
\end{minted}
\end{itemize}

\end{frame}

%----------------------------------
\begin{frame}[fragile]{Parameter Initialization}

\begin{minted}[fontsize=\footnotesize]{python}
# Lookup parameters for word embeddings
WORDS_LOOKUP = model.add_lookup_parameters((nwords, 64))

# Word-level LSTM (layers=1, input=64, hidden=128, model)
RNN = dy.LSTMBuilder(1, 64, 128, model)

# Softmax weights/biases on top of LSTM outputs
W_sm = model.add_parameters((nwords, 128))
b_sm = model.add_parameters(nwords)
\end{minted}

\end{frame}
%----------------------------------


%----------------------------------
\begin{frame}[fragile]{Sentence Initialization}

\begin{minted}[fontsize=\footnotesize]{python}
# Build the language model graph
def calc_lm_loss(wids):
    dy.renew_cg()

    # parameters -> expressions
    W_exp = dy.parameter(W_sm)
    b_exp = dy.parameter(b_sm)

    # add parameters to CG and get state
    f_init = RNN.initial_state()

    # get the word vectors for each word ID
    wembs = [WORDS_LOOKUP[wid] for wid in wids]

    # Start the rnn by inputting "<s>"
    s = f_init.add_input(wembs[-1])
\end{minted}

\end{frame}
%----------------------------------


%----------------------------------
\begin{frame}[fragile]{Loss Calculation and State Update}

\begin{minted}[fontsize=\footnotesize]{python}
    # process each word ID and embedding
    losses = []
    for wid, we in zip(wids, wembs):

        # calculate and save the softmax loss
        score = W_exp * s.output() + b_exp
        loss = dy.pickneglogsoftmax(score, wid)
        losses.append(loss)

        # update the RNN state with the input
        s = s.add_input(we)

    # return the sum of all losses
    return dy.esum(losses)
\end{minted}

\end{frame}
%----------------------------------


\begin{frame}{Custom Functions}

  \begin{itemize}
    \item DyNet has a lot of functions
      \only<2>{
        \begin{block}{Built-in Functions}
\begin{tiny}
addmv, affine\_transform, average, average\_cols, binary\_log\_loss, block\_dropout, cdiv, colwise\_add, concatenate, concatenate\_cols, const\_lookup, const\_parameter, contract3d\_1d, contract3d\_1d\_1d, conv1d\_narrow, conv1d\_wide, cube, cwise\_multiply, dot\_product, dropout, erf, exp, filter1d\_narrow, fold\_rows, hinge, huber\_distance, input, inverse, kmax\_pooling, kmh\_ngram, l1\_distance, lgamma, log, log\_softmax, logdet, logistic, logsumexp, lookup, max, min, nobackprop, noise, operator*, operator+, operator-, operator/, pairwise\_rank\_loss, parameter, pick, pickneglogsoftmax, pickrange, poisson\_loss, pow, rectify, reshape, select\_cols, select\_rows, softmax, softsign, sparsemax, sparsemax\_loss, sqrt, square, squared\_distance, squared\_norm, sum, sum\_batches, sum\_cols, tanh, trace\_of\_product, transpose, zeroes
\end{tiny}
        \end{block}
}
\only<3->{

\item Implement yourself
  \begin{itemize}
    \item Combine built-in Python operators (chain rule)
      \item Forward/Backward methods in C++
        \only<4->{\item Geometric Mean}
    \end{itemize}
}
  \end{itemize}

\end{frame}


%----------------------------------
\begin{frame}[fragile]{Forward Function}

\begin{minted}[fontsize=\footnotesize]{c}
template<class MyDevice>
void GeometricMean::forward_dev_impl(const MyDevice & dev,
            const vector<const Tensor*>& xs,
            Tensor& fx) const {
  fx.tvec().device(*dev.edevice) =
         (xs[0]->tvec() * xs[1]->tvec()).sqrt();
}
\end{minted}

\begin{itemize}
\item dev: which device (CPU/GPU)
\item xs: input values
\item fx: output value
\end{itemize}

\end{frame}
%----------------------------------


%----------------------------------
\begin{frame}[fragile]{Backward Function}

\begin{minted}[fontsize=\footnotesize]{c}
template<class MyDevice>
void GeometricMean::backward_dev_impl(const MyDevice & dev,
                  const vector<const Tensor*>& xs,
                  const Tensor& fx,
                  const Tensor& dEdf,
                  unsigned i,
                  Tensor& dEdxi) const {
  dEdxi.tvec().device(*dev.edevice) +=
         xs[i==1?0:1] * fx.inv() / 2 * dEdf;
}
\end{minted}

\begin{itemize}
\item dev: which device (CPU/GPU)
\item xs: input values
\item fx: output value
\item dEdf: derivative of loss w.r.t $f$
\item i: index of input to consider
\item dEdxi: derivative of loss w.r.t. $x[i]$
\end{itemize}

\end{frame}
%----------------------------------


\begin{frame}{Other Functions to Implement}

\begin{itemize}
\item nodes.h: class definition
\item nodes-common.cc: dimension check and function name
\item expr.h/expr.cc: interface to expressions
\item dynet.pxd/dynet.pyx: Python wrappers
\end{itemize}

\end{frame}

\begin{frame}{Wrapup}

  \begin{itemize}
    \item Rolling your own is usually not a good idea
    \item DyNet covers a very specific gap compared to TensorFlow, etc.
    \item Not just for neural models (e.g., variational objective)
      \pause
    \item Don't forget to post poject proposals!
  \end{itemize}

\end{frame}

\end{document}