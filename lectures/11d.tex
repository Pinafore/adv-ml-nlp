
\documentclass[compress]{beamer}
\usefonttheme{professionalfonts}



%

\usepackage[T1]{fontenc}
%\usepackage{kmath,kerkis}
%\usepackage{fouriernc}
\usepackage[adobe-utopia]{mathdesign}
%\usepackage{arev}
\usepackage{times}
\usepackage{natbib}

\usepackage[noend]{algpseudocode}
\usepackage{xmpmulti}
\usepackage{dsfont}
\usepackage{amsmath}

\usepackage{graphicx,float,wrapfig, bbm}
\usepackage{amsfonts, comment, bbold}
\usepackage{mdwlist}
\usepackage{subfigure}
\usepackage{colortbl}
\usepackage{mathrsfs}


\usepackage{multirow}




% packages

\usepackage{amsfonts}

% environments

\newenvironment{packed_enumerate}{
  \begin{enumerate}
    \setlength{\topsep}{0pt}
    \setlength{\itemsep}{2pt}
    \setlength{\parskip}{0pt}
    \setlength{\parsep}{0pt}
}{\end{enumerate}}

\newenvironment{stepit}
 {\begin{itemize}[<+-|alert@+>]}
   {\end{itemize}}

% commands

\newcommand{\Norm}[3]{\mathcal{N}\left( #1 \g #2, #3 \right)}
\newcommand{\popshow}[2]{\only<#1->{\alert<#1>{#2}}}
\newcommand{\x}{\mathbf{x}}
\newcommand{\ex}[1]{\mbox{exp}\left\{ #1\right\} }
\newcommand{\e}[2]{\mathbb{E}_{#1}\left[ #2 \right] }
\newcommand{\g}{\, | \,}
\newcommand{\indpt}{\protect\mathpalette{\protect\independenT}{\perp}}
\def\independenT#1#2{\mathrel{\rlap{$#1#2$}\mkern2mu{#1#2}}}
\newcommand{\E}{\textrm{E}}
\newcommand{\R}{\textrm{R}}
\newcommand{\realline}{\mathbb{R}}
\newcommand{\data}{{\cal D}}
\newcommand{\loglik}{{\cal L}}
\newcommand{\grad}[2]{ \frac{\partial{#1}}{\partial#2}}
\newcommand{\dir}[1]{\mbox{Dir}(#1)}
\newcommand{\mult}[1]{\mbox{Mult}( #1)}
\newcommand{\G}[1]{\Gamma \left( \textstyle #1 \right)}
\newcommand{\ind}[1]{\mathds{1}\left[ #1 \right] }
\newcommand{\norm}[1]{\left\lVert#1\right\rVert}

\newcommand{\class}[1]{ \texttt{#1}}
\newcommand{\term}[1]{ ``#1''}
\newcommand{\tcword}[0]{ w }
\newcommand{\docsetlabeled}[0]{ D }
\newcommand{\onedoclabeled}[0]{ d }
\newcommand{\tcposindex}[0]{ i }
\newcommand{\myblue}[1]{ {\textbf #1 }}
\newcommand{\dnrm}[1]{ _{\mbox{\textsc{ #1 }}}}
\newcommand{\argmax}[0]{ \arg \max }
\newcommand{\tcjclass}[0]{c_j}
\newcommand{\maths}[1]{ {\bf #1}}

\newcommand{\fsi}[2]{
\begin{frame}[plain]
\vspace*{-1pt}
\makebox[\linewidth]{\includegraphics[width=\paperwidth]{#1}}
\begin{center}
#2
\end{center}
\end{frame}
}





% complexity
\renewcommand{\O}{\mathcal{O}}



\setbeamersize{text margin left=0.5cm}
\setbeamersize{text margin right=0.5cm}
\setbeamercolor{alert}{fg=red!75!black}


\usetheme{default}
\useinnertheme{circles}
\useoutertheme{split}
\usecolortheme{seahorse}
% \usecolortheme{dove}
% \usecolortheme{seagull}
%\usecolortheme{default}
% \usecolortheme{dolphin}
\usefonttheme{structurebold}
%\usefonttheme{serif}

\setbeamertemplate{navigation symbols}{}
\setbeamertemplate{headline}{}
\setbeamertemplate{footline}{}
\setbeamerfont{itemize/enumerate subbody}{size=\normalsize}
\setbeamerfont{itemize/enumerate subsubbody}{size=\normalsize}
\setbeamercolor{itemize item}{fg=gray}
\setbeamercolor{enumerate item}{fg=gray}
\setbeamercolor{itemize item}{fg=gray}
\setbeamercolor{itemize subitem}{fg=gray}
\setbeamercolor{item projected}{bg=gray}
\setbeamercolor{subitem projected}{bg=gray}

\newcommand{\explain}[2]{\underbrace{#2}_{\mbox{\footnotesize{#1}}}}
\newcommand\hmmax{0}
\newcommand\bmmax{0}

\newenvironment{bullets}
{\begin{itemize} \setlength{\itemsep}{10pt}}
{\end{itemize}}

\newcommand{\mygraphic}[2]{
  \begin{beamercolorbox}[colorsep*=4pt]{black math}
    \begin{center}
      \includegraphics[#1]{#2}
    \end{center}
  \end{beamercolorbox}
}

\setbeamercolor{structure}{bg=gray}
\setbeamercolor{section in head/foot}{bg=gray}
\setbeamercolor{palette primary}{bg=lightgray}


\usepackage{minted}

\usetheme[pageofpages=of,                    % String used between the current page and the
                                             % total page count.
          bullet=circle,                     % Use circles instead of squares for bullets.
          titleline=true,                    % Show a line below the frame title.
          showdate=true,                     % show the date on the title page
          alternativetitlepage=true,         % Use the fancy title page.
          titlepagelogo=culogo,              % Logo for the first page.
          % Logo for the header on first page.
          headerlogo=boulder_cs,
          ]{UCBoulder}

\usecolortheme{ucdblack}
\author{Advanced Machine Learning for NLP}


\institute[Boyd-Graber] % (optional, but mostly needed)
{Jordan Boyd-Graber}


\AtBeginSection[] % "Beamer, do the following at the start of every section"
{ \begin{frame} \frametitle{Outline} % make a frame titled "Outline"
\tableofcontents[currentsection] % show TOC and highlight current section
\end{frame} }


\newcommand{\gfx}[2]{
\begin{center}
	\includegraphics[width=#2\linewidth]{rl/#1}
\end{center}
}
\title{Reinforcement Learning for NLP}
\date{Deep Shift-Reduce Parsers}

\usepackage{dependency/linkage6}

\begin{document}

\frame{\titlepage
\tiny
}

\begin{frame}{What Makes NLP different from RL?}

  \begin{itemize}
    \item Often, best actions are {\bf known}
    \item We're not just searching for high-reward
    \item Sometimes actions themselves are known
  \end{itemize}

\end{frame}

\begin{frame}{Roll In vs. Roll Out}
  \gfx{lols-search}{.6}
  \begin{itemize}
    \item Roll In: Which states does the algorithm see
    \item Roll Out: What states do you use for training
  \end{itemize}

\end{frame}

\begin{frame}{Known Policy vs. Exploration}

  \gfx{lols-roll}{.8}

  \begin{itemize}
    \item RL only gets reward
    \item Roll-in with reference gives unrealistic trajectories
    \item How to incorporate knowledge of true actions?
    \item Train classifier as proxy for policy
  \end{itemize}

\end{frame}

\begin{frame}{RL for Translation}

\begin{columns}
  \column{.5\linewidth}
  \gfx{nuremberg_translators}{.7}
  \column{.5\linewidth}
  \gfx{yoda}{.7}
\end{columns}
\end{frame}

\begin{frame}{RL for Translation}

\only<1>{\gfx{example_3}{.9}}
\only<2>{\gfx{example_4}{.9}}
\only<3>{\gfx{example_5}{.9}}
\only<4>{\gfx{example_6}{.9}}
\only<5>{\gfx{example_7}{.9}}
\only<6>{\gfx{example_8}{.9}}
\only<7>{\gfx{example_9}{.9}}
\only<8>{\gfx{example_10}{.9}}
\only<9>{\gfx{example_11}{.9}}
\only<10>{\gfx{example_12}{.9}}
\only<11>{\gfx{example_13}{.9}}
\only<12>{\gfx{example_14}{.9}}
\only<13>{\gfx{example_15}{.9}}
\only<14>{\gfx{example_16}{.9}}
\only<15>{\gfx{example_17}{.9}}
\only<16>{\gfx{example_18}{.9}}
\only<17>{\gfx{example_19}{.9}}


\end{frame}


\begin{frame}{How do we find a good policy?}
  \only<1>{\gfx{searn_1}{.8}}
  \only<2>{\gfx{searn_2}{.8}}
  \only<3>{\gfx{searn_3}{.8}}
  \only<4>{\gfx{searn_4}{.8}}
  \only<5>{\gfx{searn_5}{.8}}
  \only<6>{\gfx{searn_6}{.8}}
  \only<7>{\gfx{searn_7}{.8}}
  \only<8>{\gfx{searn_8}{.8}}
  \only<9>{\gfx{searn_9}{.8}}
\end{frame}


\begin{frame}{LOLS}
  \gfx{lols}{.8}
\end{frame}

\begin{frame}{LOLS on Dependency Parsing}

  Policy is learning actions for shift-reduce parser
  \gfx{lols-dp-results}{.7}

\end{frame}

\begin{frame}{But what structure is best?}

  \begin{itemize}
    \item RecNN not much better than DAN
    \item But syntax may not be optimal
    \item Can we learn structure?
      \pause
      \begin{itemize}
        \item Policy learns shift-reduce parser
        \item TreeLSTM with {\bf learned structure}
        \item Reward is performance on downstream task
      \end{itemize}
  \end{itemize}

\end{frame}

\begin{frame}{Performance}

  \gfx{learned-tree-results}{.8}

\end{frame}

\begin{frame}{What structures?}

  \gfx{learned-tree}{.8}

\end{frame}


\begin{frame}{Other places of NLP + RL}

  \begin{itemize}
    \item Question answering
    \item Language games
    \item Dialog systems
    \item Human learning
  \end{itemize}

\end{frame}

\begin{frame}{Wrapup}

  \begin{itemize}
    \item RL allows for algorithms to think about long-term rewards
    \item And to guide actions of a system
    \item Important for systems that interact with world
    \item Discrete action spaces often more difficult
  \end{itemize}

\end{frame}

\end{document}